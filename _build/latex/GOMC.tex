%% Generated by Sphinx.
\def\sphinxdocclass{report}
\documentclass[letterpaper,10pt,english]{sphinxmanual}
\ifdefined\pdfpxdimen
   \let\sphinxpxdimen\pdfpxdimen\else\newdimen\sphinxpxdimen
\fi \sphinxpxdimen=.75bp\relax

\PassOptionsToPackage{warn}{textcomp}
\usepackage[utf8]{inputenc}
\ifdefined\DeclareUnicodeCharacter
 \ifdefined\DeclareUnicodeCharacterAsOptional
  \DeclareUnicodeCharacter{"00A0}{\nobreakspace}
  \DeclareUnicodeCharacter{"2500}{\sphinxunichar{2500}}
  \DeclareUnicodeCharacter{"2502}{\sphinxunichar{2502}}
  \DeclareUnicodeCharacter{"2514}{\sphinxunichar{2514}}
  \DeclareUnicodeCharacter{"251C}{\sphinxunichar{251C}}
  \DeclareUnicodeCharacter{"2572}{\textbackslash}
 \else
  \DeclareUnicodeCharacter{00A0}{\nobreakspace}
  \DeclareUnicodeCharacter{2500}{\sphinxunichar{2500}}
  \DeclareUnicodeCharacter{2502}{\sphinxunichar{2502}}
  \DeclareUnicodeCharacter{2514}{\sphinxunichar{2514}}
  \DeclareUnicodeCharacter{251C}{\sphinxunichar{251C}}
  \DeclareUnicodeCharacter{2572}{\textbackslash}
 \fi
\fi
\usepackage{cmap}
\usepackage[T1]{fontenc}
\usepackage{amsmath,amssymb,amstext}
\usepackage{babel}
\usepackage{times}
\usepackage[Bjarne]{fncychap}
\usepackage{sphinx}

\usepackage{geometry}

% Include hyperref last.
\usepackage{hyperref}
% Fix anchor placement for figures with captions.
\usepackage{hypcap}% it must be loaded after hyperref.
% Set up styles of URL: it should be placed after hyperref.
\urlstyle{same}
\addto\captionsenglish{\renewcommand{\contentsname}{Contents:}}

\addto\captionsenglish{\renewcommand{\figurename}{Fig.}}
\addto\captionsenglish{\renewcommand{\tablename}{Table}}
\addto\captionsenglish{\renewcommand{\literalblockname}{Listing}}

\addto\captionsenglish{\renewcommand{\literalblockcontinuedname}{continued from previous page}}
\addto\captionsenglish{\renewcommand{\literalblockcontinuesname}{continues on next page}}

\addto\extrasenglish{\def\pageautorefname{page}}

\setcounter{tocdepth}{1}



\title{GOMC Documentation}
\date{Jul 01, 2018}
\release{2.31}
\author{Younes Nejahi}
\newcommand{\sphinxlogo}{\vbox{}}
\renewcommand{\releasename}{Release}
\makeindex

\begin{document}

\maketitle
\sphinxtableofcontents
\phantomsection\label{\detokenize{index::doc}}



\chapter{Overview}
\label{\detokenize{overview:overview}}\label{\detokenize{overview::doc}}
This document will instruct a new user how to download, compile, prepare the input files, and run the GOMC molecular simulation code. A basic understanding of statistical physics is recommended to complete this tutorial.

To demonstrate the capabilities of the code, the user is guided through the process of downloading, compiling a GOMC executable, and preparing input files such as PDB, PSF, Parameter, and Configuration file. Executable is then used to calculate the saturated vapor and liquid equilibria (VLE) using Gibbs Ensemble Monte Carlo on systems of pure isobutane (R600a), a branched alkane that whose application as a refrigerant/propellant is increasing. The Transferable Potentials for Phase Equilibria (TraPPE) united atom (UA) force field is used to describe the molecular geometry constraints and the intermolecular interactions.

\sphinxurl{http://en.wikipedia.org/wiki/Isobutane}


\chapter{Introduction}
\label{\detokenize{introduction:introduction}}\label{\detokenize{introduction::doc}}
GPU Optimized Monte Carlo (GOMC) is open-source software for simulating many-body molecular systems using the Metropolis Monte Carlo algorithm. GOMC is written in object oriented C++, which was chosen since it offers a good balance between code development time, interoperability with existing software elements, and code performance. The software may be compiled as a single-threaded application, a multi-threaded application using OpenMP, or to use many-core heterogeneous CPU-GPU architectures using OpenMP and CUDA. GOMC officially supports Windows 7 or newer and most modern distribution of GNU/Linux. This software has the ability to compile on recent versions of macOS; however, such a platform is not officially supported.

GOMC employs widely-used simulation file types (PDB, PSF, CHARMM-style parameter file) and supports polar and non-polar linear and branched molecules. GOMC can be used to study vapor-liquid and liquid-liquid equilibria, adsorption in porous materials, surfactant self-assembly, and condensed phase structure for complex molecules.

GOMC currently is capable of simulating systems in the following ensembles:
\begin{itemize}
\item {} 
Canonical (NVT)

\item {} 
Isobaric-isothermal (NPT)

\item {} 
Grand canonical (\(\mu\)VT)

\item {} 
Constant volume Gibbs (NVT-Gibbs)

\item {} 
Constant pressure Gibbs (NPT-Gibbs)

\end{itemize}

GOMC supports the following Monte Carlo moves:
- Rigid-body displacement
- Rigid-body rotation
- Regrowth using coupled-decoupled configurational-bias
- Intra-box swap using coupled-decoupled configurational-bias
- Inter-box swap using coupled-decoupled configurational-bias
- Volume exchange (both isotropic and anisotropic)

GOMC supports various all-atom, united-atom, and coarse grained force fields:
- OPLS
- CHARMM
- TraPPE
- Mie
- Martini


\chapter{Software Requirements}
\label{\detokenize{software_requirements:software-requirements}}\label{\detokenize{software_requirements::doc}}

\section{C++11 Compliant Compiler}
\label{\detokenize{software_requirements:c-11-compliant-compiler}}\begin{itemize}
\item {} 
Linux/macOS
\begin{itemize}
\item {} 
icpc (Intel C++ Compiler)

In Linux, the Intel compiler will generally produce the fastest CPU executables (when running on Intel Core processors). Type the following command in a terminal:

\fvset{hllines={, ,}}%
\begin{sphinxVerbatim}[commandchars=\\\{\}]
\PYGZdl{} icpc \PYGZhy{}\PYGZhy{}version
\end{sphinxVerbatim}

If gives a version number 16.0.3 (2016 Initial version) or later, you’re all set. Otherwise, we recommend upgrading.

\item {} 
g++

Type the following command in a terminal:

\fvset{hllines={, ,}}%
\begin{sphinxVerbatim}[commandchars=\\\{\}]
\PYGZdl{} g++ \PYGZhy{}\PYGZhy{}version
\end{sphinxVerbatim}

If gives a version number 4.4 or later, you’re all set. Otherwise, we recommend upgrading.

\end{itemize}

\item {} 
Windows

Visual Studio Microsoft’s Visual Studio 2010 or later is recommended.

To check the version:

\sphinxstyleemphasis{Help} (top tab) -\textgreater{} \sphinxstyleemphasis{About Microsoft Visual Studio}

\noindent\sphinxincludegraphics{{vshelp}.png}

\noindent\sphinxincludegraphics{{vsabout}.png}

\end{itemize}


\section{CMake}
\label{\detokenize{software_requirements:cmake}}
To check if cmake is installed:

\fvset{hllines={, ,}}%
\begin{sphinxVerbatim}[commandchars=\\\{\}]
\PYGZdl{} which cmake
\end{sphinxVerbatim}

To check the version number:

\fvset{hllines={, ,}}%
\begin{sphinxVerbatim}[commandchars=\\\{\}]
\PYGZdl{} cmake \PYGZhy{}\PYGZhy{}version
\end{sphinxVerbatim}

The minimum required version is 2.8. However, we recommend to use version 3.2 or later.


\section{CUDA Toolkit}
\label{\detokenize{software_requirements:cuda-toolkit}}
CUDA is required to compile the GPU executable in both Windows and Linux. However, is not required to compile the CPU code. To download and install CUDA visit NVIDIA’s webpage:

\sphinxurl{https://developer.nvidia.com/cuda-downloads}

\sphinxurl{https://developer.nvidia.com/cuda}

Please refer to CUDA Developer webpages to select an appropriate version for the desired platform. To install CUDA in Linux root/sudo, privileges are generally required. In Windows, administrative access is required.

To check if nvcc is installed:

\fvset{hllines={, ,}}%
\begin{sphinxVerbatim}[commandchars=\\\{\}]
\PYGZdl{} which nvcc
\end{sphinxVerbatim}

To check the version number:

\fvset{hllines={, ,}}%
\begin{sphinxVerbatim}[commandchars=\\\{\}]
\PYGZdl{} nvcc \PYGZhy{}\PYGZhy{}version
\end{sphinxVerbatim}

The GPU builds of the code requires NVIDIA’s CUDA 8.0 or newer.


\chapter{Recommended Software Tools}
\label{\detokenize{software_tools:recommended-software-tools}}\label{\detokenize{software_tools::doc}}
The listed programs are used in this manual and are generally considered necessary.


\section{Packmol}
\label{\detokenize{software_tools:packmol}}
Packmol is a free molecule packing tool (written in Fortran), created by José Mario Martínez, a professor of mathematics at the State University of Campinas, Brazil. Packmol allows a specified number of molecules to be packed at defined separating distances within a certain region of space. More information regarding downloading and installing Packmol is available on their homepage:

\sphinxurl{http://www.ime.unicamp.br/~martinez/packmol}

\begin{sphinxadmonition}{warning}{Warning:}
One of Packmol’s limitations is that it is unaware of topology; it treats each molecule or group of molecules as a rigid set of points. It is highly suggested to used the optimized structure of the molecule as the input file to packmol.
\end{sphinxadmonition}

\begin{sphinxadmonition}{warning}{Warning:}
Another more serious limitation is that it is not aware of periodic boundary conditions (PBC). As a result, when using Packmol to pack PDBs for GOMC, it is recommended to pack to a box 1 Angstroms smaller than the simulation box size. This prevents hard overlaps over the periodic boundary.
\end{sphinxadmonition}


\section{VMD}
\label{\detokenize{software_tools:vmd}}
VMD (Visual Molecular Dynamics) is a 3-D visualization and manipulation engine for molecular systems written in C-language. VMD is distributed and maintained by the University of Illinois at Urbana-Champaign. Its sources and binaries are free to download. It comes with a robust scripting engine, which is capable of running python and tcl scripts. More info can be found here:

\sphinxurl{http://www.ks.uiuc.edu/Research/vmd/}

Although GOMC uses the same fundamental file types, PDB (coordinates) and PSF (topology) as VMD, it uses some special tricks to obey certain rules of those file formats. One useful purpose of VMD is visualization and analyze your systems.

\begin{figure}[htbp]
\centering
\capstart

\noindent\sphinxincludegraphics{{vmd}.png}
\caption{A system of united atom isobutane molecules}\label{\detokenize{software_tools:id1}}\end{figure}

Nonetheless, the most critical part of VMD is a tool called PSFGen. PSFGen uses a tcl or python script to generate a PDB and PSF file for a system of one or more molecules. It is, perhaps, the most convenient way to generate a compliant PSF file.

\begin{figure}[htbp]
\centering
\capstart

\noindent\sphinxincludegraphics{{psfgen}.png}
\caption{An overview of the PSFGen file generation process and its relationship to VMD/NAMD}\label{\detokenize{software_tools:id2}}\end{figure}

\begin{sphinxadmonition}{tip}{Tip:}
To read more about PSFGen, reference:

\sphinxhref{http://www.ks.uiuc.edu/Research/vmd/plugins/psfgen}{Plugin homepage @ UIUC}

\sphinxhref{http://www.ks.uiuc.edu/Training/Tutorials/namd/namd-tutorial-html/node6.html}{Generating a Protein Structure File (PSF), part of the NAMD Tutorial from UIUC}

\sphinxhref{http://www.ks.uiuc.edu/Research/vmd/plugins/psfgen/ug.pdf}{In-Depth Overview {[}PDF{]}}
\end{sphinxadmonition}


\chapter{How to get the software}
\label{\detokenize{download:how-to-get-the-software}}\label{\detokenize{download::doc}}
The CPU and GPU code are merged together under GOMC project. Currently, version control is handled through the GitHub repository. The latest GOMC release, Example files, and User Manual can be downloaded from GOMC website or GitHub repository.


\section{GitHub}
\label{\detokenize{download:github}}
The posted builds in Master branch are “frozen” versions of the code that have been validated for a number of systems and ensembles. Other branches are created as a means of implementing new features. The latest updated code builds, manual, example files, and other resources can be obtained via the following GitHub repository:

\sphinxhref{https://github.com/GOMC-WSU}{GOMC GitHub Repository}

\noindent\sphinxincludegraphics{{github}.png}

GOMC and Examples repository can be found under the main page. Under GOMC repository, the code and manual can be found. Each repository can be downloaded by clicking on the Clone or download tab.

\noindent\sphinxincludegraphics{{clone}.png}


\section{Website}
\label{\detokenize{download:website}}
To access the GOMC website, please click on the following link:
\sphinxhref{http://gomc.eng.wayne.edu/}{GOMC Website}

The code can be found under the download tab, below and to the right of the logo. When new betas (or release builds) are announced, they will replace the prior code under the downloads tab. An announcement will be posted on the front page to notify users.

\noindent\sphinxincludegraphics{{website}.jpg}

GOMC is distributed as a compressed folder, containing the source and build system. To compile the code after downloading it, the first step is to extract the compressed build folder.

In Linux, the GPU and CPU codes are compressed using gzip and tar (*.tar.gz). To extract, simply move to the desire folder and type in the command line:

\fvset{hllines={, ,}}%
\begin{sphinxVerbatim}[commandchars=\\\{\}]
\PYGZdl{} tar \PYGZhy{}xzvf \PYGZlt{}file name\PYGZgt{}.tar.gz
\end{sphinxVerbatim}


\chapter{Compiling GOMC}
\label{\detokenize{compiling:compiling-gomc}}\label{\detokenize{compiling::doc}}
GOMC generates four executable files for CPU code; \sphinxcode{\sphinxupquote{GOMC\_CPU\_GEMC}} (Gibbs ensemble), \sphinxcode{\sphinxupquote{GOMC\_CPU\_NVT}} (NVT ensemble), \sphinxcode{\sphinxupquote{GOMC\_CPU\_NPT}} (isobaric-isothermal ensemble), and \sphinxcode{\sphinxupquote{GOMC\_CPU\_GCMC}} (Grand canonical ensemble). In case of installing CUDA Toolkit, GOMC will generate additional four executable files for GPU code; \sphinxcode{\sphinxupquote{GOMC\_GPU\_GEMC}}, \sphinxcode{\sphinxupquote{GOMC\_GPU\_NVT}}, \sphinxcode{\sphinxupquote{GOMC\_GPU NPT}}, and \sphinxcode{\sphinxupquote{GOMC\_GPU\_GCMC}}.

This section guid users to compile GOMC in Linux or Windows.


\section{Linux}
\label{\detokenize{compiling:linux}}
First, navigate your command line to the GOMC base directory. To compile GOMC on Linux, give permission to “metamake.sh” by running the following command and execute it:

\fvset{hllines={, ,}}%
\begin{sphinxVerbatim}[commandchars=\\\{\}]
\PYGZdl{} chmod u+x metamake.sh
\PYGZdl{} ./metamake.sh
\end{sphinxVerbatim}

This script will create a bin directory and run cmake file to compile the code as well. All executable files will be generated in the “bin” directory.


\section{Windows}
\label{\detokenize{compiling:windows}}
To compile GOMC on in Windows, follow these steps:
\begin{enumerate}
\item {} 
Open the Windows-compatible CMake GUI.

\item {} 
Set the Source Folder to the GOMC root folder.

\item {} 
Set the build Folder to your Build Folder.

\item {} 
Click configure, select your compiler/environment

\item {} 
Wait for CMake to finish the configuration.

\item {} 
Click configure again and click generate.

\item {} 
Open the CMake-generated project/solution etc. to the desired IDE (e.g Visual Studio).

\item {} 
Using the solution in the IDE of choice build GOMC per the IDE’s standard release compilation/exe- cutable generation methods.

\end{enumerate}

\begin{sphinxadmonition}{note}{Note:}
You can also use CMake from the Windows command line if its directory is added to the PATH environment variable.
\end{sphinxadmonition}


\section{Configuring CMake}
\label{\detokenize{compiling:configuring-cmake}}
GOMC uses CMAKE to generate multi-platform intermediate files to compile the project. In this section, you can find all the information needed to configure CMake.
We recommend using a different directory for the CMake output than the home directory of the project as CMake tend to generate lots of files.
CMake has a ridiculously expansive set of options, so this document will only reproduce the most obviously relevant ones. When possible, options should be passed into CMake via command line options rather than the CMakeCached.txt file:
\begin{description}
\item[{CMAKE\_BUILD\_TYPE}] \leavevmode
To get the best performance you should build the project in release mode. In CMake GUI you can set the value of “CMAKE\_BUILD\_TYPE” to “Release” and in CMake command line you can add the following to the CMake:

\fvset{hllines={, ,}}%
\begin{sphinxVerbatim}[commandchars=\\\{\}]
\PYGZhy{}DCMAKE\PYGZus{}BUILD\PYGZus{}TYPE\PYG{o}{=}Release
\end{sphinxVerbatim}

To compile the GOMC in debug mode, in CMake GUI, change the value of “CMAKE\_BUILD\_TYPE” to “Debug” and in CMake command line you can add the following to the CMake:

\fvset{hllines={, ,}}%
\begin{sphinxVerbatim}[commandchars=\\\{\}]
\PYGZhy{}DCMAKE\PYGZus{}BUILD\PYGZus{}TYPE\PYG{o}{=}Debug
\end{sphinxVerbatim}

Other options are “\textless{}None \textbar{} ReleaseWithDebInfo \textbar{} MinSizeRel\textgreater{}”.

\item[{CMAKE\_CXX\_COMPILER}] \leavevmode
This option will set the compiler. It is recommended to use the Intel Compiler and linking tools, if possible (icc/icpc/etc.). They significantly outperform the default GNU and Visual Studio compiler tools and are available for free for academic use with registration.

\item[{CMAKE\_CXX\_FLAGS\_RELEASE:STRING}] \leavevmode
To run the parallel version of CPU code, it needs to be compiled with openmp library. Open the file “CMakeCache.txt”, while still in the “bin” folder, and change the value from “-O3 -DNDEBUG” to “-O3 -qopenmp -DNDEBUG”. Recompile the GOMC by typing the command:

\fvset{hllines={, ,}}%
\begin{sphinxVerbatim}[commandchars=\\\{\}]
\PYGZdl{} make
\end{sphinxVerbatim}

\item[{ENSEMBLE\_NVT}] \leavevmode
You can turn the compilation of CPU version of NVT ensemble on or off using this option.
-DENSEMBLE\_NVT=\textless{}On \textbar{} Off\textgreater{}

\item[{ENSEMBLE\_NPT}] \leavevmode
You can turn the compilation of CPU version of NPT ensemble on or off using this option.
-DENSEMBLE\_NPT=\textless{}On \textbar{} Off\textgreater{}

\item[{ENSEMBLE\_GCMC}] \leavevmode
You can turn the compilation of CPU version of GCMC ensemble on or off using this option.
-DENSEMBLE\_GCMC=\textless{}On \textbar{} Off\textgreater{}

\item[{ENSEMBLE\_GEMC}] \leavevmode
You can turn the compilation of CPU version of GEMC ensemble on or off using this option.
-DENSEMBLE\_GEMC=\textless{}On \textbar{} Off\textgreater{}

\item[{ENSEMBLE\_GPU\_NVT}] \leavevmode
You can turn the compilation of GPU version of NVT ensemble on or off using this option.
-DENSEMBLE\_NVT=\textless{}On \textbar{} Off\textgreater{}

\item[{ENSEMBLE\_GPU\_NPT}] \leavevmode
You can turn the compilation of GPU version of NPT ensemble on or off using this option.
-DENSEMBLE\_NPT=\textless{}On \textbar{} Off\textgreater{}

\item[{ENSEMBLE\_GPU\_GCMC}] \leavevmode
You can turn the compilation of GPU version of GCMC ensemble on or off using this option.
-DENSEMBLE\_GCMC=\textless{}On \textbar{} Off\textgreater{}

\item[{ENSEMBLE\_GPU\_GEMC}] \leavevmode
You can turn the compilation of GPU version of GEMC ensemble on or off using this option.
-DENSEMBLE\_GEMC=\textless{}On \textbar{} Off\textgreater{}

\end{description}


\chapter{Input File Formats}
\label{\detokenize{input_file:input-file-formats}}\label{\detokenize{input_file::doc}}
In order to run simulation in GOMC, the following files need to be provided:
\begin{itemize}
\item {} 
GOMC executable

\item {} 
PDB file(s)

\item {} 
PSF file(s)

\item {} 
Parameter file

\item {} 
Input file “NAME.conf” (proprietary control file)

\end{itemize}


\section{PDB File}
\label{\detokenize{input_file:pdb-file}}
GOMC requires only one PDB file for NVT and NPT ensembles. However, GOMC requires two PDB files for GEMC and GCMC ensembles.


\subsection{What is PDB file}
\label{\detokenize{input_file:what-is-pdb-file}}
The term PDB can refer to the Protein Data Bank (\sphinxurl{http://www.rcsb.org/pdb/}), to a data file provided there, or to any file following the PDB format. Files in the PDB include various information such as the name of the compound, the ATOM and HETATM records containing the coordinates of the molecules, and etc. PDB widely used by NAMD, GROMACS, CHARMM, ACEMD, and Amber. GOMC ignore everything in a PDB file except for the REMARK, CRYST1, ATOM, and END records. An overview of the PDB standard can be found here:

\sphinxurl{http://www.wwpdb.org/documentation/file-format-content/format33/sect2.html\#HEADER}
\sphinxurl{http://www.wwpdb.org/documentation/file-format-content/format33/sect8.html\#CRYST1}
\sphinxurl{http://www.wwpdb.org/documentation/file-format-content/format33/sect9.html\#ATOM}

PDB contains foure major parts; \sphinxcode{\sphinxupquote{REMARK}}, \sphinxcode{\sphinxupquote{CRYST1}}, \sphinxcode{\sphinxupquote{ATOM}}, and \sphinxcode{\sphinxupquote{END}}. Here is the definition of each field and how GOMC is using them to get the information it requires.
\begin{itemize}
\item {} 
\sphinxcode{\sphinxupquote{REMARK}}:
This header records present experimental  details, annotations, comments, and information not included in other records (\sphinxurl{http://www.wwpdb.org/documentation/file-format-content/format33/} sect2.html\#HEADER). However, GOMC uses this header to print simulation informations.
\begin{itemize}
\item {} 
\sphinxstylestrong{Max Displacement} (Å)

\item {} 
\sphinxstylestrong{Max Rotation} (Degree)

\item {} 
\sphinxstylestrong{Max volume exchange} (\(Å^3\))

\item {} 
\sphinxstylestrong{Monte Carlo Steps} (MC)

\end{itemize}

\item {} 
\sphinxcode{\sphinxupquote{CRYST1}}:
This header records the unit cell dimension parameters (\sphinxurl{http://www.wwpdb.org/documentation/file-format-content/format33/sect8.html\#CRYST1}).
\begin{itemize}
\item {} 
\sphinxstylestrong{Lattice constant}: a,b,c (Å)

\item {} 
\sphinxstylestrong{Lattice angles}: \(\alpha, \beta, \gamma\) (Degree)

\end{itemize}

\item {} 
\sphinxcode{\sphinxupquote{ATOM}}:
The ATOM records present the atomic coordinates for standard amino acids and nucleotides. They also present the occupancy and temperature factor for each atom (\sphinxurl{http://www.wwpdb.org/documentation/file-format-content/format33/sect9.html\#ATOM}).
\begin{itemize}
\item {} 
\sphinxstylestrong{ATOM}: Record name

\item {} 
\sphinxstylestrong{serial}: Atom serial number.

\item {} 
\sphinxstylestrong{name}: Atom name.

\item {} 
\sphinxstylestrong{resName}: Residue name.

\item {} 
\sphinxstylestrong{chainID}: Chain identifier.

\item {} 
\sphinxstylestrong{resSeq}: Residue sequence number.

\item {} 
\sphinxstylestrong{x}: Coordinates for X (Å).

\item {} 
\sphinxstylestrong{y}: Coordinates for Y (Å).

\item {} 
\sphinxstylestrong{z}: Coordinates for Z (Å).

\item {} 
\sphinxstylestrong{occupancy}: GOMC uses to define which atoms belong to which box.

\item {} 
\sphinxstylestrong{beta}: Beta or Temperature factor. GOMC uses this value to define the mobility of the atoms. element: Element symbol.

\end{itemize}

\item {} 
\sphinxcode{\sphinxupquote{END}}:
A frame in the PDB file is terminated with the keyword.

\end{itemize}

Here are the PDB output of GOMC for the first molecule of isobutane:

\fvset{hllines={, ,}}%
\begin{sphinxVerbatim}[commandchars=\\\{\}]
REMARK       GOMC    122.790    3.14159    3439.817     1000000
CRYST1     35.245     35.245     35.245       90.00       90.00     90.00
ATOM            1         C1        ISB           1       0.911    \PYGZhy{}0.313    0.000     0.00     0.00        C
ATOM            2         C1        ISB           1       1.424    \PYGZhy{}1.765    0.000     0.00     0.00        C
ATOM            3         C1        ISB           1      \PYGZhy{}0.629    \PYGZhy{}0.313    0.000     0.00     0.00        C
ATOM            4         C1        ISB           1       1.424     0.413   \PYGZhy{}1.257     0.00     0.00        C
END
\end{sphinxVerbatim}

The fields seen here in order from left to right are the record type, atom ID, atom name, residue name, residue ID, x, y, and z coordinates, occupancy, temperature factor (called beta), and segment name.

The atom name is “C1” and residue name is “ISB”. The PSF file (next section) contains a lookup table of atoms. These contain the atom name from the PDB and the name of the atom kind in the parameter file it corresponds to. As multiple different atom names will all correspond to the same parameter, these can be viewed “atom aliases” of sorts. The chain letter (in this case ‘A’) is sometimes used when packing a number of PDBs into a single PDB file.

\begin{sphinxadmonition}{important}{Important:}\begin{itemize}
\item {} 
VMD requires a constant number of ATOMs in a multi-frame PDB (multiple records terminated by “END” in a single file). To compensate for this, all atoms from all boxes in the system are written to the output PDBs of this code.

\item {} 
For atoms not currently in a box, the coordinates are set to \sphinxcode{\sphinxupquote{\textless{} 0.00, 0.00, 0.00 \textgreater{}}}. The occupancy is commonly just set to “1.00” and is left unused by many codes. We recycle this legacy parameter by using it to denote, in our output PDBs, the box a molecule is in (box 0 occupancy=0.00 ; box 1 occupancy=1.00)

\item {} 
The beta value in GOMC code is used to define the mobility of the molecule.
\begin{itemize}
\item {} 
Beta = 0.00: molecule can move and transfer within and between boxes.

\item {} 
Beta = 1.00: molecule is fixed in its position.

\item {} 
Beta = 2.00: molecule can move within the box but cannot be transferred between boxes.

\end{itemize}

\end{itemize}
\end{sphinxadmonition}


\subsection{Generating PDB file}
\label{\detokenize{input_file:generating-pdb-file}}
With that overview of the format in mind, the following steps describe how a PDB file is typically built.
\begin{enumerate}
\item {} 
A single molecule PDB is obtained. In this example, the QM software package Gaussian was used to draw the molecule, which was then edited by hand to adhere to the PDB spec properly. The end result is a PDB for a single molecule:

\end{enumerate}
\begin{quote}

\fvset{hllines={, ,}}%
\begin{sphinxVerbatim}[commandchars=\\\{\}]
REMARK   1 File created by GaussView 5.0.8
ATOM     1  C1   ISB  1   0.911  \PYGZhy{}0.313    0.000  C
ATOM     2  C1   ISB  1   1.424  \PYGZhy{}1.765    0.000  C
ATOM     3  C1   ISB  1  \PYGZhy{}0.629  \PYGZhy{}0.313    0.000  C
ATOM     4  C1   ISB  1   1.424   0.413   \PYGZhy{}1.257  C
END
\end{sphinxVerbatim}
\end{quote}
\begin{enumerate}
\setcounter{enumi}{1}
\item {} 
Next, packings are calculated to place the simulation in a region of vapor-liquid coexistence. There are a couple of ways to do this in Gibbs ensemble:

\end{enumerate}
\begin{quote}
\begin{itemize}
\item {} 
Pack both boxes to a single middle density, which is an average of the liquid and vapor densities.

\item {} 
Same as previous method, but add a modest amount to axis of one box (e.g. 10-30 A). This technique can be handy in the constant pressure Gibbs ensemble.

\item {} 
Pack one box to the predicted liquid density and the other to the vapor density.

\end{itemize}

A good reference for getting the information needed to estimate packing is the NIST Web Book database of pure compounds:

\sphinxurl{http://webbook.nist.gov/chemistry/}
\end{quote}
\begin{enumerate}
\setcounter{enumi}{2}
\item {} 
After packing is determined, a basic pack can be performed with a Packmol script. Here is one example:

\end{enumerate}
\begin{quote}

\fvset{hllines={, ,}}%
\begin{sphinxVerbatim}[commandchars=\\\{\}]
tolerance 3.0
filetype pdb
output STEP2 ISB packed BOX 0.pdb
structure isobutane.pdb
number 1000
inside cube 0.1 0.1 0.1 70.20
end structure
\end{sphinxVerbatim}

Copy the above text into “pack\_isobutane.inp” file, save it and run the script by typing the following line into the terminal:

\fvset{hllines={, ,}}%
\begin{sphinxVerbatim}[commandchars=\\\{\}]
\PYGZdl{} ./packmol \PYGZlt{} pack\PYGZus{}isobutane.inp
\end{sphinxVerbatim}
\end{quote}


\section{PSF File}
\label{\detokenize{input_file:psf-file}}
GOMC requires only one PSF file for NVT and NPT ensembles. However, GOMC requires two PSF files for GEMC and GCMC ensembles.


\subsection{What is PSF file}
\label{\detokenize{input_file:what-is-psf-file}}
Protein structure file (PSF), contains all of the molecule-specific information needed to apply a particular force field to a molecular system. The CHARMM force field is divided into a topology file, which is needed to generate the PSF file, and a parameter file, which supplies specific numerical values for the generic CHARMM potential function. The topology file defines the atom types used in the force field; the atom names, types, bonds, and partial charges of each residue type; and any patches necessary to link or otherwise mutate these basic residues. The parameter file provides a mapping between bonded and nonbonded interactions involving the various combinations of atom types found in the topology file and specific spring constants and similar parameters for all of the bond, angle, dihedral, improper, and van der Waals terms in the CHARMM potential function. PSF file widely used by by NAMD, CHARMM, and X-PLOR.

The PSF file contains six main sections: \sphinxcode{\sphinxupquote{remarks}}, \sphinxcode{\sphinxupquote{atoms}}, \sphinxcode{\sphinxupquote{bonds}}, \sphinxcode{\sphinxupquote{angles}}, \sphinxcode{\sphinxupquote{dihedrals}}, and \sphinxcode{\sphinxupquote{impropers}} (dihedral force terms used to maintain planarity). Each section starts with a specific header described bellow:
\begin{itemize}
\item {} 
\sphinxcode{\sphinxupquote{NTITLE}}: remarks on the file.
The following is taken from a PSF file for isobutane:

\fvset{hllines={, ,}}%
\begin{sphinxVerbatim}[commandchars=\\\{\}]
PSF
      3  !NTITLE
REMARKS  original generated structure x\PYGZhy{}plor psf file
REMARKS  topology ./Top Branched Alkanes.inp
REMARKS  segment ISB \PYGZob{} first NONE; last NONE; auto angles dihedrals \PYGZcb{}
\end{sphinxVerbatim}

\item {} 
\sphinxcode{\sphinxupquote{NATOM}}: Defines the atom names, types, and partial charges of each residue type.

\fvset{hllines={, ,}}%
\begin{sphinxVerbatim}[commandchars=\\\{\}]
atom ID
segment name
residue ID
residue name
atom name
atom type
atom charge
atom mass
\end{sphinxVerbatim}

The following is taken from a PSF file for isobutane:

\fvset{hllines={, ,}}%
\begin{sphinxVerbatim}[commandchars=\\\{\}]
4000 !NATOM
1    ISB  1  ISB    C1    CH1    0.000000   13.0190  0
2    ISB  1  ISB    C2    CH3    0.000000   15.0350  0
3    ISB  1  ISB    C3    CH3    0.000000   15.0350  0
4    ISB  1  ISB    C4    CH3    0.000000   15.0350  0
5    ISB  2  ISB    C1    CH1    0.000000   13.0190  0
6    ISB  2  ISB    C2    CH3    0.000000   15.0350  0
7    ISB  2  ISB    C3    CH3    0.000000   15.0350  0
8    ISB  2  ISB    C4    CH3    0.000000   15.0350  0
\end{sphinxVerbatim}

The fields in the atom section, from left to right are atom ID, segment name, residue ID, residue name, atom name, atom type, charge, mass, and an unused 0.

\item {} 
\sphinxcode{\sphinxupquote{NBOND}}: The covalent bond section lists four pairs of atoms per line. The following is taken from a PSF file for isobutane:

\fvset{hllines={, ,}}%
\begin{sphinxVerbatim}[commandchars=\\\{\}]
3000   !BOND:     bonds
   1   2          1  3  1  4  5  6
   5   7          5  8
\end{sphinxVerbatim}

\item {} 
\sphinxcode{\sphinxupquote{NTHETA}}: The angle section lists three triples of atoms per line. The following is taken from a PSF file for isobutane:

\fvset{hllines={, ,}}%
\begin{sphinxVerbatim}[commandchars=\\\{\}]
3000   !NTHETA:   angles
   2   1          4  2  1  3  3  1  4
   6   5          8  6  5  7  7  5  8
\end{sphinxVerbatim}

\item {} 
\sphinxcode{\sphinxupquote{NPHI}}: The dihedral sections list two quadruples of atoms per line.

\item {} 
\sphinxcode{\sphinxupquote{NIMPHI}}: The improper sections list two quadruples of atoms per line. GOMC currently does not support improper. For the molecules without dihedral or improper, PDF file look like the following:

\fvset{hllines={, ,}}%
\begin{sphinxVerbatim}[commandchars=\\\{\}]
0   !NPHI: dihedrals
0   !NIMPHI: impropers
\end{sphinxVerbatim}

\item {} 
(other sections such as cross terms)

\end{itemize}

\begin{sphinxadmonition}{important}{Important:}\begin{itemize}
\item {} 
The PSF file format is a highly redundant file format. It repeats identical topology of thousands of molecules of a common kind in some cases. GOMC follows the same approach as NAMD, allowing this excess information externally and compiling it in the code.

\item {} 
Other sections (e.g. cross terms) contain unsupported or legacy parameters and are ignored.

\item {} 
Following the restrictions of VMD, the order of the PSF atoms must match the order in the.

\item {} 
Improper entries are read and stored, but are not currently used. Support will eventually be added for this.

\end{itemize}
\end{sphinxadmonition}


\subsection{Generating PSF file}
\label{\detokenize{input_file:generating-psf-file}}
The PSF file is typically generated using PSFGen. It is convenient to make a script, such as the example below, to do this:

Typically, one script is run per box to generate a finalized PDB/PSF for that box. The script requires one additional file, the NAMD-style topology file. While GOMC does not directly read or interact with this file, it’s typically used to generate the PSF and, hence, is considered one of the integral file types. It will be briefly discussed in the following section.


\section{Topology File}
\label{\detokenize{input_file:topology-file}}
A CHARMM forcefield topology file contains all of the information needed to convert a list of residue names into a complete PSF structure file. The topology is a whitespace separated file format, which contains a list of atoms and their corresponding masses, and a list of residue information (charges, composition, and topology). Essentially, it is a non-redundant lookup table equivalent to the PSF file.

This is followed by a series of residues, which tell PSFGen what atoms are bonded to a given atom. Each residue is comprised of four key elements:
\begin{itemize}
\item {} 
A header beginning with the keyword RESI with the residue name and net charge

\item {} 
A body with multiple ATOM entries (not to be confused with the PDB-style entries of the same name), which list the partial charge on the particle and what kind of atom each named atom in a specific molecule/residue is.

\item {} 
A section of lines starting with the word BOND contains pairs of bonded atoms (typically 3 per line)

\item {} 
A closing section with instructions for PSFGen.

\end{itemize}

Here’s an example of topology file for isobutane:

\fvset{hllines={, ,}}%
\begin{sphinxVerbatim}[commandchars=\\\{\}]
* Custom top file \PYGZhy{}\PYGZhy{} branched alkanes *
11
!
MASS 1 CH3 15.035 C !
MASS 2 CH1 13.019 C !

AUTOGENERATE ANGLES DIHEDRALS

RESI ISB    0.00 !  isobutane \PYGZhy{} TraPPE
GROUP
ATOM C1 CH1 0.00 !  C3
ATOM C2 CH3 0.00 !  C2\PYGZhy{}C1
ATOM C3 CH3 0.00 !  C4
ATOM C4 CH3 0.00 !
BOND C1 C2 C1 C3 C1 C4
PATCHING FIRS NONE LAST NONE

END
\end{sphinxVerbatim}

\begin{sphinxadmonition}{note}{Note:}
The keyword END must be used to terminate this file and keywords related to the auto-generation process must be placed near the top of the file, after the MASS definitions.
\end{sphinxadmonition}

\begin{sphinxadmonition}{tip}{Tip:}
More in-depth information can be found in the following links:
\begin{itemize}
\item {} 
\sphinxhref{http://www.ks.uiuc.edu/Training/Tutorials/science/topology/topology-tutorial.pdf}{Topology Tutorial}

\end{itemize}
\begin{itemize}
\item {} 
\sphinxhref{http://www.ks.uiuc.edu/Training/Tutorials/science/topology/topology-html/node4.html}{NAMD Tutorial: Examining the Topology File}

\end{itemize}
\begin{itemize}
\item {} 
\sphinxhref{http://www.ks.uiuc.edu/Training/Tutorials/science/forcefield-tutorial/forcefield-html/node6.html}{Developing Topology and Parameter Files}

\end{itemize}
\begin{itemize}
\item {} 
\sphinxhref{http://www.ks.uiuc.edu/Training/Tutorials/namd/namd-tutorial-win-html/node25.html}{NAMD Tutorial: Topology Files}

\end{itemize}
\end{sphinxadmonition}


\section{Parameter File(s)}
\label{\detokenize{input_file:parameter-file-s}}
Currently, GOMC uses a single parameter file and the user has the two kinds of parameter file choices:
\begin{itemize}
\item {} 
\sphinxcode{\sphinxupquote{CHARMM}} (Chemistry at Harvard Molecular Mechanics) compatible parameter file

\item {} 
\sphinxcode{\sphinxupquote{EXOTIC}} parameter file

\end{itemize}

If the parameter file type is not specified or if the chosen file is missing, an error will result.

Both force field file options are whitespace separated files with sections preceded by a tag. When a known tag (representing a molecular interaction in the model) is encountered, reading of that section of the force field begins. Comments (anything after a \sphinxcode{\sphinxupquote{*}} or \sphinxcode{\sphinxupquote{!}}) and whitespace are ignored. Reading concludes when the end of the file is reached or another section tag is encountered.


\subsection{CHARMM format parameter file}
\label{\detokenize{input_file:charmm-format-parameter-file}}
CHARMM contains a widely used model for describing energies in Monte Carlo and molecular dynamics simulations. It is intended to be compatible with other codes that use such a format, such as NAMD. See \sphinxhref{http://www.charmmtutorial.org/index.php/The\_Energy\_Function}{here} for a general overview of the CHARMM force field.

Here’s the basic CHARMM contributions that are supported in GOMC:
\begin{equation*}
\begin{split}U_{\texttt{bond}}&=\sum_{\texttt{bonds}} K_b(b-b_0)^2\\
U_{\texttt{dihedral}}&=\sum_{\texttt{dihedrals}} K_{\phi} [1+\cos(n\phi - \delta)]\\
U_{\texttt{angle}}&=\sum_{\texttt{angles}} K_{\theta}(\theta-\theta_0)^2\\
U_{\texttt{LJ}}&=\sum_{\texttt{nonbonded}} \epsilon_{ij}\left[\left(\frac{R_{min_{ij}}}{r_{ij}}\right)^{12}-2\left(\frac{R_{min_{ij}}}{r_{ij}}\right)^6\right]+ \frac{q_i q_j}{\epsilon r_{ij}} \\\end{split}
\end{equation*}
As seen above, the following are recognized, read and used:
\begin{itemize}
\item {} 
\sphinxcode{\sphinxupquote{BONDS}}
- Quadratic expression describing bond stretching based on bond length (b) in Angstrom
\textendash{} Typically, it is ignored as bonds are rigid for Monte Carlo simulations. To specify that it is to be ignored, put a very large value i.e. “999999999999” for \(K_b\).

\begin{sphinxadmonition}{note}{Note:}
GOMC does not sample bond stretch.
\end{sphinxadmonition}

\begin{figure}[htbp]
\centering
\capstart

\noindent\sphinxincludegraphics{{bonds}.png}
\caption{Oscillations about the equilibrium bond length}\label{\detokenize{input_file:id1}}\end{figure}

\item {} 
\sphinxcode{\sphinxupquote{ANGLES}}
- Describe the conformational lbehavior of an angle (\(\delta\)) between three atoms, one of which is shared branch point to the other two. To fix any angle and ignore the related angle energy, put a very large value i.e. “999999999999” for \(K_\delta\).

\begin{figure}[htbp]
\centering
\capstart

\noindent\sphinxincludegraphics{{angle}.png}
\caption{Oscillations of 3 atoms about an equilibrium bond angle}\label{\detokenize{input_file:id2}}\end{figure}

\item {} 
\sphinxcode{\sphinxupquote{DIHEDRALS}}
- Describes crankshaft-like rotation behavior about a central bond in a series of three consecutive bonds (rotation is given as \(\phi\)).

\begin{figure}[htbp]
\centering
\capstart

\noindent\sphinxincludegraphics{{dihedrals}.png}
\caption{Torsional rotation of 4 atoms about a central bond}\label{\detokenize{input_file:id3}}\end{figure}

\item {} 
\sphinxcode{\sphinxupquote{NONBONDED}}
- This tag name only should be used if CHARMM force files are being used. This section describes 12-6 (Lennard-Jones) non-bonded interactions. Non-bonded parameters are assigned by specifying atom type name followed by polarizabilities (which will be ignored), minimum energy, and (minimum radius)/2. In order to modify 1-4 interaction, a second polarizability (again, will be ignored), minimum energy, and (minimum radius)/2 need to be defined; otherwise, the same parameter will be considered for 1-4 interaction.

\begin{figure}[htbp]
\centering
\capstart

\noindent\sphinxincludegraphics{{nonbonded}.png}
\caption{Non-bonded energy terms (electrostatics and Lennard-Jones)}\label{\detokenize{input_file:id4}}\end{figure}

\item {} 
\sphinxcode{\sphinxupquote{NBFIX}}
- This tag name only should be used if CHARMM force field is being used. This section allows in- teraction between two pairs of atoms to be modified, done by specifying two atom type names followed by minimum energy and minimum radius. In order to modify 1-4 interaction, a second minimum energy and minimum radius need to be defined; otherwise, the same parameter will be considered for 1-4 interaction.

\begin{sphinxadmonition}{note}{Note:}
Please pay attention that in this section we define minimum radius, not (minimum ra- dius)/2 as it is defined in the NONBONDED section.
\end{sphinxadmonition}

Currently, supported sections of the \sphinxcode{\sphinxupquote{CHARMM}} compliant file include \sphinxcode{\sphinxupquote{BONDS}}, \sphinxcode{\sphinxupquote{ANGLES}}, \sphinxcode{\sphinxupquote{DIHEDRALS}}, \sphinxcode{\sphinxupquote{NONBONDED}}, \sphinxcode{\sphinxupquote{NBFIX}}. Other sections such as \sphinxcode{\sphinxupquote{CMAP}} are not currently read or supported.

\end{itemize}


\subsection{BONDS}
\label{\detokenize{input_file:bonds}}
(“bond stretching”) is one key section of the CHARMM-compliant file. Units for the \(K_b\) variable in this section are in kcal/mol; the \(b_0\) section (which represents the equilibrium bond length for that kind of pair) is measured in Angstroms.

\fvset{hllines={, ,}}%
\begin{sphinxVerbatim}[commandchars=\\\{\}]
BONDS
!V(bond) = Kb(b \PYGZhy{} b0)**2
!
!Kb:  kcal/mole/A**2
!b0:  A
!
!  Kb (kcal/mol) = Kb (K) * Boltz.  const.;
!
!atom type Kb b0 description
CH3 CH1 9999999999 1.540 !  TraPPE 2
\end{sphinxVerbatim}

\begin{sphinxadmonition}{note}{Note:}
The \(K_b\) value may appear odd, but this is because a larger value corresponds to a more rigid bond. As Monte Carlo force fields (e.g. TraPPE) typically treat molecules as rigid constructs, \(K_b\) is set to a large value - 9999999999. Sampling bond stretch is not supported in GOMC.
\end{sphinxadmonition}


\subsection{ANGLES}
\label{\detokenize{input_file:angles}}
(“bond bending”), where \(\theta\) and \(\theta_0\) are commonly measured in degrees and \(K_\theta\) is measured in kcal/mol/K. These values, in literature, are often expressed in Kelvin (K). To convert Kelvin to kcal/mol/K, multiply by the Boltzmann constant \textendash{} \(K_\theta\), 0.0019872041 kcal/mol. In order to fix the angle, it requires to set a large value for \(K_\theta\). By assigning a large value like 9999999999, specified angle will be fixed and energy of that angle will considered to be zero.

Here is an example of what is necessary for isobutane:

\fvset{hllines={, ,}}%
\begin{sphinxVerbatim}[commandchars=\\\{\}]
ANGLES
!
!V(angle) = Ktheta(Theta \PYGZhy{} Theta0)**2
!
!V(Urey\PYGZhy{}Bradley) = Kub(S \PYGZhy{} S0)**2
!
!Ktheta:  kcal/mole/rad**2
!Theta0:  degrees
!S0:  A
!
!  Ktheta (kcal/mol) = Ktheta (K) * Boltz.  const.
!
!atom types Ktheta Theta0 Kub(?)  S0(?)
CH3 CH1 CH3 62.100125 112.00 !  TraPPE 2
\end{sphinxVerbatim}

Some CHARMM ANGLES section entries include \sphinxcode{\sphinxupquote{Urey-Bradley}} potentials (\(K_{ub}\), \(b_{ub}\)), in addition to the standard quadratic angle potential. The constants related to this potential function are currently read, but the logic has not been added to calculate this potential function. Support for this potential function will be added in later versions of the code.


\subsection{DIHEDRALS}
\label{\detokenize{input_file:dihedrals}}
The final major bonded interactions section of the CHARMM compliant parameter file are the DIHEDRALS. Each dihedral is composed of a dihedral series of 1 or more terms. Often, there are 4 to 6 terms in a dihedral. Angles for the dihedrals’ deltas are given in degrees.

Since isobutane has no dihedral, here are the parameters pertaining to 2,3-dimethylbutane:

\fvset{hllines={, ,}}%
\begin{sphinxVerbatim}[commandchars=\\\{\}]
DIHEDRALS
!
!V(dihedral) = Kchi(1 + cos(n(chi) \PYGZhy{} delta))
!
!Kchi:  kcal/mole
!n:  multiplicity
!delta:  degrees
!
!  Kchi (kcal/mol) = Kchi (K) * Boltz.  const.
!
!atom types Kchi n delta description
X CH1 CH1 X \PYGZhy{}0.498907 0 0.0   !  TraPPE 2
X CH1 CH1 X  0.851974 1 0.0   !  TraPPE 2
X CH1 CH1 X \PYGZhy{}0.222269 2 180.0 !  TraPPE 2
X CH1 CH1 X  0.876894 3 0.0   !  TraPPE 2
\end{sphinxVerbatim}

\begin{sphinxadmonition}{note}{Note:}
The code allows the use of ‘X’ to indicate ambiguous positions on the ends. This is useful because this kind is often determined solely by the two middle atoms in the middle of the dihedral, according to literature.
\end{sphinxadmonition}


\subsection{IMPROPERS}
\label{\detokenize{input_file:impropers}}
Energy parameters used to describe out-of-plane rocking are currently read, but unused. The section is often blank. If it becomes necessary, algorithms to calculate the improper energy will need to be added.


\subsection{NONBONDED}
\label{\detokenize{input_file:nonbonded}}
The next section of the CHARMM style parameter file is the NONBONDED. In order to use TraPPE this section of the CHARMM compliant file is critical. Here’s an example with our isobutane potential model:

\fvset{hllines={, ,}}%
\begin{sphinxVerbatim}[commandchars=\\\{\}]
NONBONDED
!
!V(Lennard\PYGZhy{}Jones) = Eps,i,j[(Rmin,i,j/ri,j)**12 \PYGZhy{} 2(Rmin,i,j/ri,j)**6]
!
!atom ignored epsilon Rmin/2 ignored eps,1\PYGZhy{}4 Rmin/2,1\PYGZhy{}4
!
CH3 0.0 \PYGZhy{}0.194745992 2.10461634058 0.0 0.0 0.0 !  TraPPE 1
CH1 0.0 \PYGZhy{}0.019872040 2.62656119304 0.0 0.0 0.0 !  TraPPE 2
End
\end{sphinxVerbatim}

\begin{sphinxadmonition}{note}{Note:}
The \(R_{min}\) is different from \(\sigma\). \(\sigma\) is the distance to the x-intercept (where interaction energy goes from being repulsive to positive). \(R_{min}\) is the potential well-depth, where the attraction is maximum. To convert \(\sigma\) to \(R_{min}\), simply multiply \(\sigma\) by 0.56123102415, and flag it with a negative sign.
\end{sphinxadmonition}


\subsection{NBFIX}
\label{\detokenize{input_file:nbfix}}
The last section of the CHARMM style parameter file is the NBFIX. In this section, individual pair interaction will be modified. First, pseudo non-bonded parameters have to be defined in NONBONDED and modified in NBFIX. Here?s an example if it is required to modify interaction between CH3 and CH1 atoms:

\fvset{hllines={, ,}}%
\begin{sphinxVerbatim}[commandchars=\\\{\}]
NBFIX
!V(Lennard\PYGZhy{}Jones) = Eps,i,j[(Rmin,i,j/ri,j)**12 \PYGZhy{} 2(Rmin,i,j/ri,j)**6]
!
!atom atom epsilon Rmin eps,1\PYGZhy{}4 Rmin,1\PYGZhy{}4
CH3 CH1 \PYGZhy{}0.294745992 1.10461634058 !
End
\end{sphinxVerbatim}


\section{Exotic Parameter File}
\label{\detokenize{input_file:exotic-parameter-file}}
The exotic file is intended for use with nonstandard/specialty models of molecular interaction, which are not included in CHARMM standard. Currently, two custom interaction are included:
\begin{itemize}
\item {} 
\sphinxcode{\sphinxupquote{NONBODED\_MIE}} This section describes n-6 (Lennard-Jones) non-bonded interactions. The Lennard- Jones potential (12-6) is a subset of this potential. Non-bonded parameters are assigned by specifying atom type name followed by minimum energy, atom diameter, and repulsion exponent. In order to modify 1-4 interaction, a second minimum energy, atom diameter, and repulsion exponent need to be defined; otherwise, the same parameters would be considered for 1-4 interaction.

\item {} 
\sphinxcode{\sphinxupquote{NBFIX\_MIE}} This section allows n-6 (Lennard-Jones) interaction between two pairs of atoms to be mod- ified. This is done by specifying two atoms type names followed by minimum energy, atom diameter, and repulsion exponent. In order to modify 1-4 interaction, a second minimum energy, atom diameter, and repulsion exponent need to be defined; otherwise, the same parameter will be considered for 1-4 interaction.

\end{itemize}

\begin{sphinxadmonition}{note}{Note:}
In \sphinxcode{\sphinxupquote{EXOTIC}} force field, the definition of atom diameter(\(\sigma\)) is same for both NONBONDED\_MIE and NBFIX\_MIE.
\end{sphinxadmonition}

Otherwise, the exotic file reuses the same geometry section headings - BONDS / ANGLES / DIHEDRALS / etc. The only difference in these sections versus in the CHARMM format force field file is that the energies are in Kelvin (‘K’), the unit most commonly found for parameters in Monte Carlo chemical simulation literature. This precludes the need to convert to kcal/mol, the energy unit used in CHARMM.
The most frequently used section of the exotic files in the Mie potential section is NONBONDED\_MIE. Here are the parameters that are used to simulate alkanes:

\fvset{hllines={, ,}}%
\begin{sphinxVerbatim}[commandchars=\\\{\}]
NONBONDED\PYGZus{}MIE
!
!V(mie) = 4*eps*((sig ij/r ij)\PYGZca{}n\PYGZhy{}(sig ij/r ij)\PYGZca{}6)
!
!atom eps sig n eps,1\PYGZhy{}4 sig,1\PYGZhy{}4 n,1\PYGZhy{}4
CH4 161.00 3.740 14 0.0 0.0 0.0 ! Potoff, et al. \PYGZsq{}09
CH3 121.25 3.783 16 0.0 0.0 0.0 ! Potoff, et al. \PYGZsq{}09
CH2  61.00 3.990 16 0.0 0.0 0.0 ! Potoff, et al. \PYGZsq{}09
\end{sphinxVerbatim}

\begin{sphinxadmonition}{note}{Note:}
Although the units (Angstroms) are the same, the exotic file uses \(\sigma\), not the \(R_{min}\) used by CHARMM. The energy in the exotic file are expressed in Kelvin (K), as this is the standard convention in the literature.
\end{sphinxadmonition}


\section{Control File (*.conf)}
\label{\detokenize{input_file:control-file-conf}}
The control file is GOMC’s proprietary input file. It contains key settings. The settings generally fall under three categories:
\begin{itemize}
\item {} 
Input/Simulation Setup

\item {} 
System Settings for During Run

\item {} 
Output Settings

\end{itemize}

\begin{sphinxadmonition}{note}{Note:}
The control file is designed to recognize logic values, such as “yes/true/on” or “no/false/off”.
\end{sphinxadmonition}


\subsection{Input/Simulation Setup}
\label{\detokenize{input_file:input-simulation-setup}}
In this section, input file names are listed. In addition, if you want to restart your simulation or use integer seed for running your simulation, you need to modify this section according to your purpose.
\begin{description}
\item[{\sphinxcode{\sphinxupquote{Restart}}}] \leavevmode
Determines whether to restart the simulation from previous simulation or not.
\begin{itemize}
\item {} 
Value 1: Boolean - True if restart, false otherwise.

\end{itemize}

\item[{\sphinxcode{\sphinxupquote{PRNG}}}] \leavevmode
Dictates how to start the pseudo-random number generator (PRNG)
\begin{itemize}
\item {} 
Value 1: String
\begin{itemize}
\item {} 
RANDOM: Randomizes Mersenne Twister PRNG with random bits based on the system time.

\end{itemize}

\fvset{hllines={, ,}}%
\begin{sphinxVerbatim}[commandchars=\\\{\}]
\PYGZsh{}\PYGZsh{}\PYGZsh{}\PYGZsh{}\PYGZsh{}\PYGZsh{}\PYGZsh{}\PYGZsh{}\PYGZsh{}\PYGZsh{}\PYGZsh{}\PYGZsh{}\PYGZsh{}\PYGZsh{}\PYGZsh{}\PYGZsh{}\PYGZsh{}\PYGZsh{}\PYGZsh{}\PYGZsh{}\PYGZsh{}\PYGZsh{}\PYGZsh{}\PYGZsh{}\PYGZsh{}\PYGZsh{}\PYGZsh{}\PYGZsh{}\PYGZsh{}\PYGZsh{}\PYGZsh{}\PYGZsh{}\PYGZsh{}
\PYGZsh{} kind \PYGZob{}RANDOM, INTSEED\PYGZcb{}
\PYGZsh{}\PYGZsh{}\PYGZsh{}\PYGZsh{}\PYGZsh{}\PYGZsh{}\PYGZsh{}\PYGZsh{}\PYGZsh{}\PYGZsh{}\PYGZsh{}\PYGZsh{}\PYGZsh{}\PYGZsh{}\PYGZsh{}\PYGZsh{}\PYGZsh{}\PYGZsh{}\PYGZsh{}\PYGZsh{}\PYGZsh{}\PYGZsh{}\PYGZsh{}\PYGZsh{}\PYGZsh{}\PYGZsh{}\PYGZsh{}\PYGZsh{}\PYGZsh{}\PYGZsh{}\PYGZsh{}\PYGZsh{}\PYGZsh{}
PRNG RANDOM
\end{sphinxVerbatim}
\begin{itemize}
\item {} 
INTSEED: This option “seeds” the Mersenne Twister PRNG with a standard integer. When the same integer is used, the generated PRNG stream should be the same every time, which is helpful in tracking down bugs.

\end{itemize}

\end{itemize}

\item[{\sphinxcode{\sphinxupquote{Random\_Seed}}}] \leavevmode
Defines the seed number. If “INTSEED” is chosen, seed number needs to be specified; otherwise, the program will terminate.
\begin{itemize}
\item {} 
Value 1: ULONG - If “INTSEED” option is selected for PRNG (See above example)

\end{itemize}

\fvset{hllines={, ,}}%
\begin{sphinxVerbatim}[commandchars=\\\{\}]
\PYGZsh{}\PYGZsh{}\PYGZsh{}\PYGZsh{}\PYGZsh{}\PYGZsh{}\PYGZsh{}\PYGZsh{}\PYGZsh{}\PYGZsh{}\PYGZsh{}\PYGZsh{}\PYGZsh{}\PYGZsh{}\PYGZsh{}\PYGZsh{}\PYGZsh{}\PYGZsh{}\PYGZsh{}\PYGZsh{}\PYGZsh{}\PYGZsh{}\PYGZsh{}\PYGZsh{}\PYGZsh{}\PYGZsh{}\PYGZsh{}\PYGZsh{}\PYGZsh{}\PYGZsh{}\PYGZsh{}\PYGZsh{}\PYGZsh{}
\PYGZsh{} kind \PYGZob{}RANDOM, INTSEED\PYGZcb{}
\PYGZsh{}\PYGZsh{}\PYGZsh{}\PYGZsh{}\PYGZsh{}\PYGZsh{}\PYGZsh{}\PYGZsh{}\PYGZsh{}\PYGZsh{}\PYGZsh{}\PYGZsh{}\PYGZsh{}\PYGZsh{}\PYGZsh{}\PYGZsh{}\PYGZsh{}\PYGZsh{}\PYGZsh{}\PYGZsh{}\PYGZsh{}\PYGZsh{}\PYGZsh{}\PYGZsh{}\PYGZsh{}\PYGZsh{}\PYGZsh{}\PYGZsh{}\PYGZsh{}\PYGZsh{}\PYGZsh{}\PYGZsh{}\PYGZsh{}
PRNG INTSEED
Random Seed 50
\end{sphinxVerbatim}

\item[{\sphinxcode{\sphinxupquote{ParaTypeCHARMM}}}] \leavevmode
Sets force field type to CHARMM style.
\begin{itemize}
\item {} 
Value 1: Boolean - True if it is CHARMM forcefield, false otherwise.

\end{itemize}

\fvset{hllines={, ,}}%
\begin{sphinxVerbatim}[commandchars=\\\{\}]
\PYGZsh{}\PYGZsh{}\PYGZsh{}\PYGZsh{}\PYGZsh{}\PYGZsh{}\PYGZsh{}\PYGZsh{}\PYGZsh{}\PYGZsh{}\PYGZsh{}\PYGZsh{}\PYGZsh{}\PYGZsh{}\PYGZsh{}\PYGZsh{}\PYGZsh{}\PYGZsh{}\PYGZsh{}\PYGZsh{}\PYGZsh{}\PYGZsh{}\PYGZsh{}\PYGZsh{}\PYGZsh{}\PYGZsh{}\PYGZsh{}\PYGZsh{}\PYGZsh{}\PYGZsh{}\PYGZsh{}\PYGZsh{}\PYGZsh{}
\PYGZsh{} FORCE FIELD TYPE
\PYGZsh{}\PYGZsh{}\PYGZsh{}\PYGZsh{}\PYGZsh{}\PYGZsh{}\PYGZsh{}\PYGZsh{}\PYGZsh{}\PYGZsh{}\PYGZsh{}\PYGZsh{}\PYGZsh{}\PYGZsh{}\PYGZsh{}\PYGZsh{}\PYGZsh{}\PYGZsh{}\PYGZsh{}\PYGZsh{}\PYGZsh{}\PYGZsh{}\PYGZsh{}\PYGZsh{}\PYGZsh{}\PYGZsh{}\PYGZsh{}\PYGZsh{}\PYGZsh{}\PYGZsh{}\PYGZsh{}\PYGZsh{}\PYGZsh{}
ParaTypeCHARMM true
\end{sphinxVerbatim}

\item[{\sphinxcode{\sphinxupquote{ParaTypeEXOTIC}}}] \leavevmode
Sets force field type to EXOTIC style.
\begin{itemize}
\item {} 
Value 1: Boolean - True if it is EXOTIC forcefield, false otherwise.

\end{itemize}

\fvset{hllines={, ,}}%
\begin{sphinxVerbatim}[commandchars=\\\{\}]
\PYGZsh{}\PYGZsh{}\PYGZsh{}\PYGZsh{}\PYGZsh{}\PYGZsh{}\PYGZsh{}\PYGZsh{}\PYGZsh{}\PYGZsh{}\PYGZsh{}\PYGZsh{}\PYGZsh{}\PYGZsh{}\PYGZsh{}\PYGZsh{}\PYGZsh{}\PYGZsh{}\PYGZsh{}\PYGZsh{}\PYGZsh{}\PYGZsh{}\PYGZsh{}\PYGZsh{}\PYGZsh{}\PYGZsh{}\PYGZsh{}\PYGZsh{}\PYGZsh{}\PYGZsh{}\PYGZsh{}\PYGZsh{}\PYGZsh{}
\PYGZsh{} FORCE FIELD TYPE
\PYGZsh{}\PYGZsh{}\PYGZsh{}\PYGZsh{}\PYGZsh{}\PYGZsh{}\PYGZsh{}\PYGZsh{}\PYGZsh{}\PYGZsh{}\PYGZsh{}\PYGZsh{}\PYGZsh{}\PYGZsh{}\PYGZsh{}\PYGZsh{}\PYGZsh{}\PYGZsh{}\PYGZsh{}\PYGZsh{}\PYGZsh{}\PYGZsh{}\PYGZsh{}\PYGZsh{}\PYGZsh{}\PYGZsh{}\PYGZsh{}\PYGZsh{}\PYGZsh{}\PYGZsh{}\PYGZsh{}\PYGZsh{}\PYGZsh{}
ParaTypeEXOTIC true
\end{sphinxVerbatim}

\item[{\sphinxcode{\sphinxupquote{ParaTypeMARTINI}}}] \leavevmode
Sets force field type to MARTINI style.
\begin{itemize}
\item {} 
Value 1: Boolean - True if it is MARTINI forcefield, false otherwise.

\end{itemize}

\fvset{hllines={, ,}}%
\begin{sphinxVerbatim}[commandchars=\\\{\}]
\PYGZsh{}\PYGZsh{}\PYGZsh{}\PYGZsh{}\PYGZsh{}\PYGZsh{}\PYGZsh{}\PYGZsh{}\PYGZsh{}\PYGZsh{}\PYGZsh{}\PYGZsh{}\PYGZsh{}\PYGZsh{}\PYGZsh{}\PYGZsh{}\PYGZsh{}\PYGZsh{}\PYGZsh{}\PYGZsh{}\PYGZsh{}\PYGZsh{}\PYGZsh{}\PYGZsh{}\PYGZsh{}\PYGZsh{}\PYGZsh{}\PYGZsh{}\PYGZsh{}\PYGZsh{}\PYGZsh{}\PYGZsh{}\PYGZsh{}
\PYGZsh{} FORCE FIELD TYPE
\PYGZsh{}\PYGZsh{}\PYGZsh{}\PYGZsh{}\PYGZsh{}\PYGZsh{}\PYGZsh{}\PYGZsh{}\PYGZsh{}\PYGZsh{}\PYGZsh{}\PYGZsh{}\PYGZsh{}\PYGZsh{}\PYGZsh{}\PYGZsh{}\PYGZsh{}\PYGZsh{}\PYGZsh{}\PYGZsh{}\PYGZsh{}\PYGZsh{}\PYGZsh{}\PYGZsh{}\PYGZsh{}\PYGZsh{}\PYGZsh{}\PYGZsh{}\PYGZsh{}\PYGZsh{}\PYGZsh{}\PYGZsh{}\PYGZsh{}
ParaTypeMARTINI true
\end{sphinxVerbatim}

\item[{\sphinxcode{\sphinxupquote{Parameters}}}] \leavevmode
Provides the name and location of the parameter file to use for the simulation.
\begin{itemize}
\item {} 
Value 1: String - Sets the name of the parameter file.

\end{itemize}

\fvset{hllines={, ,}}%
\begin{sphinxVerbatim}[commandchars=\\\{\}]
\PYGZsh{}\PYGZsh{}\PYGZsh{}\PYGZsh{}\PYGZsh{}\PYGZsh{}\PYGZsh{}\PYGZsh{}\PYGZsh{}\PYGZsh{}\PYGZsh{}\PYGZsh{}\PYGZsh{}\PYGZsh{}\PYGZsh{}\PYGZsh{}\PYGZsh{}\PYGZsh{}\PYGZsh{}\PYGZsh{}\PYGZsh{}\PYGZsh{}\PYGZsh{}\PYGZsh{}\PYGZsh{}\PYGZsh{}\PYGZsh{}\PYGZsh{}\PYGZsh{}\PYGZsh{}\PYGZsh{}\PYGZsh{}\PYGZsh{}
\PYGZsh{} FORCE FIELD TYPE
\PYGZsh{}\PYGZsh{}\PYGZsh{}\PYGZsh{}\PYGZsh{}\PYGZsh{}\PYGZsh{}\PYGZsh{}\PYGZsh{}\PYGZsh{}\PYGZsh{}\PYGZsh{}\PYGZsh{}\PYGZsh{}\PYGZsh{}\PYGZsh{}\PYGZsh{}\PYGZsh{}\PYGZsh{}\PYGZsh{}\PYGZsh{}\PYGZsh{}\PYGZsh{}\PYGZsh{}\PYGZsh{}\PYGZsh{}\PYGZsh{}\PYGZsh{}\PYGZsh{}\PYGZsh{}\PYGZsh{}\PYGZsh{}\PYGZsh{}
ParaTypeCHARMM yes
Parameters ../../common/Par\PYGZus{}TraPPE\PYGZus{}Alkanes.inp
\end{sphinxVerbatim}

\item[{\sphinxcode{\sphinxupquote{Coordinates}}}] \leavevmode
Defines the PDB file names (coordinates) and location for each box in the system.
\begin{itemize}
\item {} 
Value 1: Integer - Sets box number (starts from ‘0’).

\item {} 
Value 2: String - Sets the name of PDB file.

\end{itemize}

\begin{sphinxadmonition}{note}{Note:}
NVT and NPT ensembles requires only one PDB file and GEMC/GCMC requires two PDB files. If the number of PDB files is not compatible with the simulation type, the program will terminate.
\end{sphinxadmonition}

Example of NVT or NPT ensemble:

\fvset{hllines={, ,}}%
\begin{sphinxVerbatim}[commandchars=\\\{\}]
\PYGZsh{}\PYGZsh{}\PYGZsh{}\PYGZsh{}\PYGZsh{}\PYGZsh{}\PYGZsh{}\PYGZsh{}\PYGZsh{}\PYGZsh{}\PYGZsh{}\PYGZsh{}\PYGZsh{}\PYGZsh{}\PYGZsh{}\PYGZsh{}\PYGZsh{}\PYGZsh{}\PYGZsh{}\PYGZsh{}\PYGZsh{}\PYGZsh{}\PYGZsh{}\PYGZsh{}\PYGZsh{}\PYGZsh{}\PYGZsh{}\PYGZsh{}\PYGZsh{}\PYGZsh{}\PYGZsh{}\PYGZsh{}\PYGZsh{}
\PYGZsh{} INPUT PDB FILES \PYGZhy{} NVT or NPT ensemble
\PYGZsh{}\PYGZsh{}\PYGZsh{}\PYGZsh{}\PYGZsh{}\PYGZsh{}\PYGZsh{}\PYGZsh{}\PYGZsh{}\PYGZsh{}\PYGZsh{}\PYGZsh{}\PYGZsh{}\PYGZsh{}\PYGZsh{}\PYGZsh{}\PYGZsh{}\PYGZsh{}\PYGZsh{}\PYGZsh{}\PYGZsh{}\PYGZsh{}\PYGZsh{}\PYGZsh{}\PYGZsh{}\PYGZsh{}\PYGZsh{}\PYGZsh{}\PYGZsh{}\PYGZsh{}\PYGZsh{}\PYGZsh{}\PYGZsh{}
Coordinates 0 STEP3\PYGZus{}START\PYGZus{}ISB\PYGZus{}sys.pdb
\end{sphinxVerbatim}

Example of Gibbs or GC ensemble:

\fvset{hllines={, ,}}%
\begin{sphinxVerbatim}[commandchars=\\\{\}]
\PYGZsh{}\PYGZsh{}\PYGZsh{}\PYGZsh{}\PYGZsh{}\PYGZsh{}\PYGZsh{}\PYGZsh{}\PYGZsh{}\PYGZsh{}\PYGZsh{}\PYGZsh{}\PYGZsh{}\PYGZsh{}\PYGZsh{}\PYGZsh{}\PYGZsh{}\PYGZsh{}\PYGZsh{}\PYGZsh{}\PYGZsh{}\PYGZsh{}\PYGZsh{}\PYGZsh{}\PYGZsh{}\PYGZsh{}\PYGZsh{}\PYGZsh{}\PYGZsh{}\PYGZsh{}\PYGZsh{}\PYGZsh{}\PYGZsh{}
\PYGZsh{} INPUT PDB FILES \PYGZhy{} Gibbs or GC ensemble
\PYGZsh{}\PYGZsh{}\PYGZsh{}\PYGZsh{}\PYGZsh{}\PYGZsh{}\PYGZsh{}\PYGZsh{}\PYGZsh{}\PYGZsh{}\PYGZsh{}\PYGZsh{}\PYGZsh{}\PYGZsh{}\PYGZsh{}\PYGZsh{}\PYGZsh{}\PYGZsh{}\PYGZsh{}\PYGZsh{}\PYGZsh{}\PYGZsh{}\PYGZsh{}\PYGZsh{}\PYGZsh{}\PYGZsh{}\PYGZsh{}\PYGZsh{}\PYGZsh{}\PYGZsh{}\PYGZsh{}\PYGZsh{}\PYGZsh{}
Coordinates 0 STEP3\PYGZus{}START\PYGZus{}ISB\PYGZus{}sys\PYGZus{}BOX\PYGZus{}0.pdb
Coordinates 1 STEP3\PYGZus{}START\PYGZus{}ISB\PYGZus{}sys\PYGZus{}BOX\PYGZus{}1.pdb
\end{sphinxVerbatim}

\begin{sphinxadmonition}{note}{Note:}
In case of Restart true, the restart PDB output file from GOMC (OutputName\_BOX\_0\_restart.pdb) can be used for each box.
\end{sphinxadmonition}

Example of Gibbs ensemble when Restart mode is active:

\fvset{hllines={, ,}}%
\begin{sphinxVerbatim}[commandchars=\\\{\}]
\PYGZsh{}\PYGZsh{}\PYGZsh{}\PYGZsh{}\PYGZsh{}\PYGZsh{}\PYGZsh{}\PYGZsh{}\PYGZsh{}\PYGZsh{}\PYGZsh{}\PYGZsh{}\PYGZsh{}\PYGZsh{}\PYGZsh{}\PYGZsh{}\PYGZsh{}\PYGZsh{}\PYGZsh{}\PYGZsh{}\PYGZsh{}\PYGZsh{}\PYGZsh{}\PYGZsh{}\PYGZsh{}\PYGZsh{}\PYGZsh{}\PYGZsh{}\PYGZsh{}\PYGZsh{}\PYGZsh{}\PYGZsh{}\PYGZsh{}
\PYGZsh{} INPUT PDB FILES
\PYGZsh{}\PYGZsh{}\PYGZsh{}\PYGZsh{}\PYGZsh{}\PYGZsh{}\PYGZsh{}\PYGZsh{}\PYGZsh{}\PYGZsh{}\PYGZsh{}\PYGZsh{}\PYGZsh{}\PYGZsh{}\PYGZsh{}\PYGZsh{}\PYGZsh{}\PYGZsh{}\PYGZsh{}\PYGZsh{}\PYGZsh{}\PYGZsh{}\PYGZsh{}\PYGZsh{}\PYGZsh{}\PYGZsh{}\PYGZsh{}\PYGZsh{}\PYGZsh{}\PYGZsh{}\PYGZsh{}\PYGZsh{}\PYGZsh{}
Coordinates 0 ISB\PYGZus{}T\PYGZus{}270\PYGZus{}k\PYGZus{}BOX\PYGZus{}0\PYGZus{}restart.pdb
Coordinates 1 ISB\PYGZus{}T\PYGZus{}270\PYGZus{}k\PYGZus{}BOX\PYGZus{}1\PYGZus{}restart.pdb
\end{sphinxVerbatim}

\item[{\sphinxcode{\sphinxupquote{Structures}}}] \leavevmode
Defines the PSF filenames (structures) for each box in the system.
\begin{itemize}
\item {} 
Value 1: Integer - Sets box number (start from ‘0’)

\item {} 
Value 2: String - Sets the name of PSF file.

\end{itemize}

\begin{sphinxadmonition}{note}{Note:}
NVT and NPT ensembles requires only one PSF file and GEMC/GCMC requires two PSF files. If the number of PSF files is not compatible with the simulation type, the program will terminate.
\end{sphinxadmonition}

Example of NVT or NPT ensemble:

\fvset{hllines={, ,}}%
\begin{sphinxVerbatim}[commandchars=\\\{\}]
\PYGZsh{}\PYGZsh{}\PYGZsh{}\PYGZsh{}\PYGZsh{}\PYGZsh{}\PYGZsh{}\PYGZsh{}\PYGZsh{}\PYGZsh{}\PYGZsh{}\PYGZsh{}\PYGZsh{}\PYGZsh{}\PYGZsh{}\PYGZsh{}\PYGZsh{}\PYGZsh{}\PYGZsh{}\PYGZsh{}\PYGZsh{}\PYGZsh{}\PYGZsh{}\PYGZsh{}\PYGZsh{}\PYGZsh{}\PYGZsh{}\PYGZsh{}\PYGZsh{}\PYGZsh{}\PYGZsh{}\PYGZsh{}\PYGZsh{}
\PYGZsh{} INPUT PSF FILES
\PYGZsh{}\PYGZsh{}\PYGZsh{}\PYGZsh{}\PYGZsh{}\PYGZsh{}\PYGZsh{}\PYGZsh{}\PYGZsh{}\PYGZsh{}\PYGZsh{}\PYGZsh{}\PYGZsh{}\PYGZsh{}\PYGZsh{}\PYGZsh{}\PYGZsh{}\PYGZsh{}\PYGZsh{}\PYGZsh{}\PYGZsh{}\PYGZsh{}\PYGZsh{}\PYGZsh{}\PYGZsh{}\PYGZsh{}\PYGZsh{}\PYGZsh{}\PYGZsh{}\PYGZsh{}\PYGZsh{}\PYGZsh{}\PYGZsh{}
Structure 0 STEP3\PYGZus{}START\PYGZus{}ISB\PYGZus{}sys.psf
\end{sphinxVerbatim}

Example of Gibbs or GC ensemble:

\fvset{hllines={, ,}}%
\begin{sphinxVerbatim}[commandchars=\\\{\}]
\PYGZsh{}\PYGZsh{}\PYGZsh{}\PYGZsh{}\PYGZsh{}\PYGZsh{}\PYGZsh{}\PYGZsh{}\PYGZsh{}\PYGZsh{}\PYGZsh{}\PYGZsh{}\PYGZsh{}\PYGZsh{}\PYGZsh{}\PYGZsh{}\PYGZsh{}\PYGZsh{}\PYGZsh{}\PYGZsh{}\PYGZsh{}\PYGZsh{}\PYGZsh{}\PYGZsh{}\PYGZsh{}\PYGZsh{}\PYGZsh{}\PYGZsh{}\PYGZsh{}\PYGZsh{}\PYGZsh{}\PYGZsh{}\PYGZsh{}
\PYGZsh{} INPUT PSF FILES
\PYGZsh{}\PYGZsh{}\PYGZsh{}\PYGZsh{}\PYGZsh{}\PYGZsh{}\PYGZsh{}\PYGZsh{}\PYGZsh{}\PYGZsh{}\PYGZsh{}\PYGZsh{}\PYGZsh{}\PYGZsh{}\PYGZsh{}\PYGZsh{}\PYGZsh{}\PYGZsh{}\PYGZsh{}\PYGZsh{}\PYGZsh{}\PYGZsh{}\PYGZsh{}\PYGZsh{}\PYGZsh{}\PYGZsh{}\PYGZsh{}\PYGZsh{}\PYGZsh{}\PYGZsh{}\PYGZsh{}\PYGZsh{}\PYGZsh{}
Structure 0 STEP3\PYGZus{}START\PYGZus{}ISB\PYGZus{}sys\PYGZus{}BOX\PYGZus{}0.psf
Structure 1 STEP3\PYGZus{}START\PYGZus{}ISB\PYGZus{}sys\PYGZus{}BOX\PYGZus{}1.psf
\end{sphinxVerbatim}

\begin{sphinxadmonition}{note}{Note:}
In case of Restart true, the PSF output file from GOMC (OutputName merged.psf) can be used for both boxes.
\end{sphinxadmonition}

Example of Gibbs ensemble when Restart mode is active:

\fvset{hllines={, ,}}%
\begin{sphinxVerbatim}[commandchars=\\\{\}]
\PYGZsh{}\PYGZsh{}\PYGZsh{}\PYGZsh{}\PYGZsh{}\PYGZsh{}\PYGZsh{}\PYGZsh{}\PYGZsh{}\PYGZsh{}\PYGZsh{}\PYGZsh{}\PYGZsh{}\PYGZsh{}\PYGZsh{}\PYGZsh{}\PYGZsh{}\PYGZsh{}\PYGZsh{}\PYGZsh{}\PYGZsh{}\PYGZsh{}\PYGZsh{}\PYGZsh{}\PYGZsh{}\PYGZsh{}\PYGZsh{}\PYGZsh{}\PYGZsh{}\PYGZsh{}\PYGZsh{}\PYGZsh{}\PYGZsh{}
\PYGZsh{} INPUT PSF FILES
\PYGZsh{}\PYGZsh{}\PYGZsh{}\PYGZsh{}\PYGZsh{}\PYGZsh{}\PYGZsh{}\PYGZsh{}\PYGZsh{}\PYGZsh{}\PYGZsh{}\PYGZsh{}\PYGZsh{}\PYGZsh{}\PYGZsh{}\PYGZsh{}\PYGZsh{}\PYGZsh{}\PYGZsh{}\PYGZsh{}\PYGZsh{}\PYGZsh{}\PYGZsh{}\PYGZsh{}\PYGZsh{}\PYGZsh{}\PYGZsh{}\PYGZsh{}\PYGZsh{}\PYGZsh{}\PYGZsh{}\PYGZsh{}\PYGZsh{}
Structure 0 ISB\PYGZus{}T\PYGZus{}270\PYGZus{}k\PYGZus{}merged.psf
Structure 1 ISB\PYGZus{}T\PYGZus{}270\PYGZus{}k\PYGZus{}merged.psf
\end{sphinxVerbatim}

\end{description}


\subsection{System Settings for During Run Setup}
\label{\detokenize{input_file:system-settings-for-during-run-setup}}
This section contains all the variables not involved in the output of data during the simulation, or in the reading of input files at the start of the simulation. In other words, it contains settings related to the moves, the thermodynamic constants (based on choice of ensemble), and the length of the simulation.
Note that some tags, or entries for tags, are only used in certain ensembles (e.g. Gibbs ensemble). These cases are denoted with colored text.
\begin{description}
\item[{\sphinxcode{\sphinxupquote{GEMC}}}] \leavevmode
\sphinxstyleemphasis{(For Gibbs Ensemble runs only)} Defines the type of Gibbs Ensemble simulation you want to run. If neglected in Gibbs Ensemble, it simply defaults to const volume (NVT) Gibbs Ensemble.
\begin{itemize}
\item {} 
Value 1: String - Allows you to pick between isovolumetric (“NVT”) and isobaric (“NPT”) Gibbs ensemble simulations.

\end{itemize}

\fvset{hllines={, ,}}%
\begin{sphinxVerbatim}[commandchars=\\\{\}]
\PYGZsh{}\PYGZsh{}\PYGZsh{}\PYGZsh{}\PYGZsh{}\PYGZsh{}\PYGZsh{}\PYGZsh{}\PYGZsh{}\PYGZsh{}\PYGZsh{}\PYGZsh{}\PYGZsh{}\PYGZsh{}\PYGZsh{}\PYGZsh{}\PYGZsh{}\PYGZsh{}\PYGZsh{}\PYGZsh{}\PYGZsh{}\PYGZsh{}\PYGZsh{}\PYGZsh{}\PYGZsh{}\PYGZsh{}\PYGZsh{}\PYGZsh{}\PYGZsh{}\PYGZsh{}\PYGZsh{}\PYGZsh{}\PYGZsh{}
\PYGZsh{} GEMC TYPE (DEFAULT IS NVT GEMC)
\PYGZsh{}\PYGZsh{}\PYGZsh{}\PYGZsh{}\PYGZsh{}\PYGZsh{}\PYGZsh{}\PYGZsh{}\PYGZsh{}\PYGZsh{}\PYGZsh{}\PYGZsh{}\PYGZsh{}\PYGZsh{}\PYGZsh{}\PYGZsh{}\PYGZsh{}\PYGZsh{}\PYGZsh{}\PYGZsh{}\PYGZsh{}\PYGZsh{}\PYGZsh{}\PYGZsh{}\PYGZsh{}\PYGZsh{}\PYGZsh{}\PYGZsh{}\PYGZsh{}\PYGZsh{}\PYGZsh{}\PYGZsh{}\PYGZsh{}
GEMC NVT
\end{sphinxVerbatim}

\item[{\sphinxcode{\sphinxupquote{Pressure}}}] \leavevmode
If “NPT” simulation is chosen, imposed pressure (in bar) needs to be specified; otherwise, the program will terminate.
\begin{itemize}
\item {} 
Value 1: Double - Constant pressure in bars.

\end{itemize}

\fvset{hllines={, ,}}%
\begin{sphinxVerbatim}[commandchars=\\\{\}]
\PYGZsh{}\PYGZsh{}\PYGZsh{}\PYGZsh{}\PYGZsh{}\PYGZsh{}\PYGZsh{}\PYGZsh{}\PYGZsh{}\PYGZsh{}\PYGZsh{}\PYGZsh{}\PYGZsh{}\PYGZsh{}\PYGZsh{}\PYGZsh{}\PYGZsh{}\PYGZsh{}\PYGZsh{}\PYGZsh{}\PYGZsh{}\PYGZsh{}\PYGZsh{}\PYGZsh{}\PYGZsh{}\PYGZsh{}\PYGZsh{}\PYGZsh{}\PYGZsh{}\PYGZsh{}\PYGZsh{}\PYGZsh{}\PYGZsh{}
\PYGZsh{} GEMC TYPE (DEFAULT IS NVT GEMC)
\PYGZsh{}\PYGZsh{}\PYGZsh{}\PYGZsh{}\PYGZsh{}\PYGZsh{}\PYGZsh{}\PYGZsh{}\PYGZsh{}\PYGZsh{}\PYGZsh{}\PYGZsh{}\PYGZsh{}\PYGZsh{}\PYGZsh{}\PYGZsh{}\PYGZsh{}\PYGZsh{}\PYGZsh{}\PYGZsh{}\PYGZsh{}\PYGZsh{}\PYGZsh{}\PYGZsh{}\PYGZsh{}\PYGZsh{}\PYGZsh{}\PYGZsh{}\PYGZsh{}\PYGZsh{}\PYGZsh{}\PYGZsh{}\PYGZsh{}
GEMC NPT
Pressure 5.76
\end{sphinxVerbatim}

\item[{\sphinxcode{\sphinxupquote{Temperature}}}] \leavevmode
Sets the temperature at which the system will run.
\begin{itemize}
\item {} 
Value 1: Double - Constant temperature of simulation in degrees Kelvin.

\end{itemize}

\item[{\sphinxcode{\sphinxupquote{Rcut}}}] \leavevmode
Sets a specific radius that non-bonded interaction energy and force will be considered and calculated using defined potential function.
\begin{itemize}
\item {} 
Value 1: Double - The distance to truncate the Lennard-Jones potential at.

\end{itemize}

\item[{\sphinxcode{\sphinxupquote{RcutLow}}}] \leavevmode
Sets a specific minimum possible in angstrom that reject any move that place any atom closer than specified distance.
\begin{itemize}
\item {} 
Value 1: Double - The minimum possible distance between any atoms.

\end{itemize}

\item[{\sphinxcode{\sphinxupquote{LRC}}}] \leavevmode
Defines whether or not long range corrections are used.
\begin{itemize}
\item {} 
Value 1: Boolean - True to consider long range correction. In case of using “SHIFT” or “SWITCH” potential functions, LRC will be ignored.

\end{itemize}

\item[{\sphinxcode{\sphinxupquote{Exclude}}}] \leavevmode
Defines which pairs of bonded atoms should be excluded from non-bonded interactions.
\begin{itemize}
\item {} 
Value 1: String - Allows you to choose between “1-2”, “1-3”, and “1-4”.
\begin{description}
\item[{1-2}] \leavevmode
All interactions pairs of bonded atoms, except the ones that separated with one bond, will be considered and modified using 1-4 parameters defined in parameter file.

\item[{1-3}] \leavevmode
All interaction pairs of bonded atoms, except the ones that separated with one or two bonds, will be considered and modified using 1-4 parameters defined in parameter file.

\item[{1-4}] \leavevmode
All interaction pairs of bonded atoms, except the ones that separated with one, two or three bonds, will be considered using non-bonded parameters defined in parameter file.

\end{description}

\begin{sphinxadmonition}{note}{Note:}
The default value is “1-4”.
\end{sphinxadmonition}

\begin{sphinxadmonition}{note}{Note:}
In CHARMM force field, the 1-4 interaction needs to be considered. Choosing “Exclude 1-3” will modify 1-4 interaction based on 1-4 parameter in parameter file. If a kind force field is used, where 1-4 interaction needs to be ignored, such as TraPPE, either “exclude 1-4” needs to be chosen or 1-4 parameter needs to be assigned a value of zero in the parameter file.
\end{sphinxadmonition}

\end{itemize}

\item[{\sphinxcode{\sphinxupquote{Potential}}}] \leavevmode
Defines the potential function type to calculate non-bonded interaction energy and force between atoms.
\begin{itemize}
\item {} 
Value 1: String - Allows you to pick between “VDW”, “SHIFT” and “SWITCH”.
\begin{description}
\item[{VDW}] \leavevmode
Nonbonded interaction energy and force calculated based on n-6 (Lennard-Johns) equation. This function will be discussed further in the Intermolecular energy and Virial calculation section.

\fvset{hllines={, ,}}%
\begin{sphinxVerbatim}[commandchars=\\\{\}]
\PYGZsh{}\PYGZsh{}\PYGZsh{}\PYGZsh{}\PYGZsh{}\PYGZsh{}\PYGZsh{}\PYGZsh{}\PYGZsh{}\PYGZsh{}\PYGZsh{}\PYGZsh{}\PYGZsh{}\PYGZsh{}\PYGZsh{}\PYGZsh{}\PYGZsh{}\PYGZsh{}\PYGZsh{}\PYGZsh{}\PYGZsh{}\PYGZsh{}\PYGZsh{}\PYGZsh{}\PYGZsh{}\PYGZsh{}\PYGZsh{}\PYGZsh{}\PYGZsh{}\PYGZsh{}\PYGZsh{}\PYGZsh{}\PYGZsh{}
\PYGZsh{} SIMULATION CONDITION
\PYGZsh{}\PYGZsh{}\PYGZsh{}\PYGZsh{}\PYGZsh{}\PYGZsh{}\PYGZsh{}\PYGZsh{}\PYGZsh{}\PYGZsh{}\PYGZsh{}\PYGZsh{}\PYGZsh{}\PYGZsh{}\PYGZsh{}\PYGZsh{}\PYGZsh{}\PYGZsh{}\PYGZsh{}\PYGZsh{}\PYGZsh{}\PYGZsh{}\PYGZsh{}\PYGZsh{}\PYGZsh{}\PYGZsh{}\PYGZsh{}\PYGZsh{}\PYGZsh{}\PYGZsh{}\PYGZsh{}\PYGZsh{}\PYGZsh{}
Temperature 270.00
Potential VDW
LRC true
Rcut 10
Exclude 1\PYGZhy{}4
\end{sphinxVerbatim}

\item[{SHIFT}] \leavevmode
This option forces the potential energy to be zero at Rcut distance. This function will be discussed further in the Intermolecular energy and Virial calculation section.

\fvset{hllines={, ,}}%
\begin{sphinxVerbatim}[commandchars=\\\{\}]
\PYGZsh{}\PYGZsh{}\PYGZsh{}\PYGZsh{}\PYGZsh{}\PYGZsh{}\PYGZsh{}\PYGZsh{}\PYGZsh{}\PYGZsh{}\PYGZsh{}\PYGZsh{}\PYGZsh{}\PYGZsh{}\PYGZsh{}\PYGZsh{}\PYGZsh{}\PYGZsh{}\PYGZsh{}\PYGZsh{}\PYGZsh{}\PYGZsh{}\PYGZsh{}\PYGZsh{}\PYGZsh{}\PYGZsh{}\PYGZsh{}\PYGZsh{}\PYGZsh{}\PYGZsh{}\PYGZsh{}\PYGZsh{}\PYGZsh{}
\PYGZsh{} SIMULATION CONDITION
\PYGZsh{}\PYGZsh{}\PYGZsh{}\PYGZsh{}\PYGZsh{}\PYGZsh{}\PYGZsh{}\PYGZsh{}\PYGZsh{}\PYGZsh{}\PYGZsh{}\PYGZsh{}\PYGZsh{}\PYGZsh{}\PYGZsh{}\PYGZsh{}\PYGZsh{}\PYGZsh{}\PYGZsh{}\PYGZsh{}\PYGZsh{}\PYGZsh{}\PYGZsh{}\PYGZsh{}\PYGZsh{}\PYGZsh{}\PYGZsh{}\PYGZsh{}\PYGZsh{}\PYGZsh{}\PYGZsh{}\PYGZsh{}\PYGZsh{}
Temperature 270.00
Potential SHIFT
LRC false
Rcut 10
Exclude 1\PYGZhy{}4
\end{sphinxVerbatim}

\item[{SWITCH}] \leavevmode
This option smoothly forces the potential energy to be zero at Rcut distance and starts modifying the potential at Rswitch distance. Depending on force field type, specific potential function will be applied. These functions will be discussed further in the Intermolecular energy and Virial calculation section.

\end{description}

\end{itemize}

\item[{\sphinxcode{\sphinxupquote{Rswitch}}}] \leavevmode
In the case of choosing “SWITCH” as potential function, a distance is set in which non-bonded interaction energy is truncated smoothly from to cutoff distance.
\begin{itemize}
\item {} 
Value 1: Double - Define switch distance in angstrom. If the “SWITCH” function is chosen, Rswitch needs to be defined; otherwise, the program will be terminated.

\end{itemize}

\item[{\sphinxcode{\sphinxupquote{ElectroStatic}}}] \leavevmode
Considers coulomb interaction or not. This function will be discussed further in the Inter- molecular energy and Virial calculation section.
\begin{itemize}
\item {} 
Value 1: Boolean - True if coulomb interaction needs to be considered and false if not.

\begin{sphinxadmonition}{note}{Note:}
If MARTINI force field was used and charged molecule was used in simulation, ElectroStatic needs to be turn on. MARTINI force field uses short range coulomb interaction with constant dielectric 15.0.
\end{sphinxadmonition}

\end{itemize}

\item[{\sphinxcode{\sphinxupquote{Ewald}}}] \leavevmode
Considers standard Ewald summation method for electrostatic calculation. This function will be discussed further in the Intermolecular energy and Virial calculation section.
\begin{itemize}
\item {} 
Value 1: Double - True if Ewald summation calculation needs to be considered and false if not.

\begin{sphinxadmonition}{note}{Note:}
By default, \sphinxcode{\sphinxupquote{ElectroStatic}} will be set to true if Ewald summation method was used to calculate coulomb interaction.
\end{sphinxadmonition}

\end{itemize}

\item[{\sphinxcode{\sphinxupquote{CachedFourier}}}] \leavevmode
Considers storing the reciprocal terms for Ewald summation calculation in order to improve the code performance. This option would increase the code performance with the cost of memory usage.
\begin{itemize}
\item {} 
Value 1: Boolean - True to store reciprocal terms of Ewald summation calculation and false if not.

\begin{sphinxadmonition}{note}{Note:}
By default, CachedFourier will be set to true if not value was set.
\end{sphinxadmonition}

\end{itemize}

\item[{\sphinxcode{\sphinxupquote{Tolerance}}}] \leavevmode
Specifies the accuracy of the Ewald summation calculation. Ewald separation parameter and number of reciprocal vectors for the Ewald summation are determined based on the accuracy parameter.
\begin{itemize}
\item {} 
Value 1: Double - Sets the accuracy in Ewald summation calculation. A reasonable value for te accuracy is 0.00001.

\begin{sphinxadmonition}{note}{Note:}
If “Ewald” was chosen and no value was set for Tolerance, the program will be terminated.
\end{sphinxadmonition}

\end{itemize}

\item[{\sphinxcode{\sphinxupquote{Dielectric}}}] \leavevmode
Defines dielectric constant for coulomb interaction in MARTINI force field.
\begin{itemize}
\item {} 
Value 1: Double - Sets dielectric value used in coulomb interaction.

\begin{sphinxadmonition}{note}{Note:}
In MARTINI force field, Dielectric needs to be set to 15.0. If MARTINI force field was chosen and if Dielectric was not specified, a default value of 15.0 will be assigned.
\end{sphinxadmonition}

\end{itemize}

\item[{\sphinxcode{\sphinxupquote{PressureCalc}}}] \leavevmode
Considers to calculate the pressure or not. If it is set to true, the frequency of pressure calculation need to be set.
\begin{itemize}
\item {} 
Value 1: Boolean - True enabling pressure calculation during the simulation, false disabling pressure calculation.

\item {} 
Value 2: Ulong - The frequency of calculating the pressure.

\end{itemize}

\item[{\sphinxcode{\sphinxupquote{1-4scaling}}}] \leavevmode
Defines constant factor to modify intra-molecule coulomb interaction.
\begin{itemize}
\item {} 
Value 1: Double - A fraction number between 0.0 and 1.0.

\begin{sphinxadmonition}{note}{Note:}
CHARMM force field uses a value between 0.0 and 1.0. In MARTINI force field, it needs to be set to 1.0 because 1-4 interaction will not be modified in this force field.
\end{sphinxadmonition}

\fvset{hllines={, ,}}%
\begin{sphinxVerbatim}[commandchars=\\\{\}]
\PYGZsh{}\PYGZsh{}\PYGZsh{}\PYGZsh{}\PYGZsh{}\PYGZsh{}\PYGZsh{}\PYGZsh{}\PYGZsh{}\PYGZsh{}\PYGZsh{}\PYGZsh{}\PYGZsh{}\PYGZsh{}\PYGZsh{}\PYGZsh{}\PYGZsh{}\PYGZsh{}\PYGZsh{}\PYGZsh{}\PYGZsh{}\PYGZsh{}\PYGZsh{}\PYGZsh{}\PYGZsh{}\PYGZsh{}\PYGZsh{}\PYGZsh{}\PYGZsh{}\PYGZsh{}\PYGZsh{}\PYGZsh{}\PYGZsh{}
\PYGZsh{} SIMULATION CONDITION
\PYGZsh{}\PYGZsh{}\PYGZsh{}\PYGZsh{}\PYGZsh{}\PYGZsh{}\PYGZsh{}\PYGZsh{}\PYGZsh{}\PYGZsh{}\PYGZsh{}\PYGZsh{}\PYGZsh{}\PYGZsh{}\PYGZsh{}\PYGZsh{}\PYGZsh{}\PYGZsh{}\PYGZsh{}\PYGZsh{}\PYGZsh{}\PYGZsh{}\PYGZsh{}\PYGZsh{}\PYGZsh{}\PYGZsh{}\PYGZsh{}\PYGZsh{}\PYGZsh{}\PYGZsh{}\PYGZsh{}\PYGZsh{}\PYGZsh{}
ElectroStatic true
Ewald true
Tolerance 0.00001
1\PYGZhy{}4scaling 0.0
\end{sphinxVerbatim}

\end{itemize}

\item[{\sphinxcode{\sphinxupquote{RunSteps}}}] \leavevmode
Sets the total number of steps to run (one move is performed for each step) (cycles = this value / number of molecules in the system)
\begin{itemize}
\item {} 
Value 1: Ulong - Total run steps

\end{itemize}

\item[{\sphinxcode{\sphinxupquote{EqSteps}}}] \leavevmode
Sets the number of steps necessary to equilibrate the system; averaging will begin at this step.
\begin{itemize}
\item {} 
Value 1: Ulong - Equilibration steps

\end{itemize}

\item[{\sphinxcode{\sphinxupquote{AdjSteps}}}] \leavevmode
Sets the number of steps per adjustment to the maximum constants associated with each move (e.g. maximum distance in xyz to displace, the maximum volume in \(Å^3\) to swap, etc.)
\begin{itemize}
\item {} 
Value 1: Ulong - Number of steps per move adjustment

\fvset{hllines={, ,}}%
\begin{sphinxVerbatim}[commandchars=\\\{\}]
\PYGZsh{}\PYGZsh{}\PYGZsh{}\PYGZsh{}\PYGZsh{}\PYGZsh{}\PYGZsh{}\PYGZsh{}\PYGZsh{}\PYGZsh{}\PYGZsh{}\PYGZsh{}\PYGZsh{}\PYGZsh{}\PYGZsh{}\PYGZsh{}\PYGZsh{}\PYGZsh{}\PYGZsh{}\PYGZsh{}\PYGZsh{}\PYGZsh{}\PYGZsh{}\PYGZsh{}\PYGZsh{}\PYGZsh{}\PYGZsh{}\PYGZsh{}\PYGZsh{}\PYGZsh{}\PYGZsh{}\PYGZsh{}\PYGZsh{}
\PYGZsh{} STEPS
\PYGZsh{}\PYGZsh{}\PYGZsh{}\PYGZsh{}\PYGZsh{}\PYGZsh{}\PYGZsh{}\PYGZsh{}\PYGZsh{}\PYGZsh{}\PYGZsh{}\PYGZsh{}\PYGZsh{}\PYGZsh{}\PYGZsh{}\PYGZsh{}\PYGZsh{}\PYGZsh{}\PYGZsh{}\PYGZsh{}\PYGZsh{}\PYGZsh{}\PYGZsh{}\PYGZsh{}\PYGZsh{}\PYGZsh{}\PYGZsh{}\PYGZsh{}\PYGZsh{}\PYGZsh{}\PYGZsh{}\PYGZsh{}\PYGZsh{}
RunSteps 25000000
EqSteps 5000000
AdjSteps 1000
\end{sphinxVerbatim}

\end{itemize}

\item[{\sphinxcode{\sphinxupquote{ChemPot}}}] \leavevmode
For Grand Canonical (GC) ensemble runs only: Chemical potential at which simulation is run.
\begin{itemize}
\item {} 
Value 1: String - The resname to apply this chemical potential.

\item {} 
Value 2: Double - The chemical potential value in degrees Kelvin (should be negative).

\end{itemize}

\begin{sphinxadmonition}{note}{Note:}
For binary systems, include multiple copies of the tag (one per residue kind).
\end{sphinxadmonition}

\begin{sphinxadmonition}{note}{Note:}
If there is a molecule kind that cannot be transfer between boxes (in PDB file the beta value is set to 1.00 or 2.00), an arbitrary value (e.g. 0.00) can be assigned to the resname.
\end{sphinxadmonition}

\fvset{hllines={, ,}}%
\begin{sphinxVerbatim}[commandchars=\\\{\}]
\PYGZsh{}\PYGZsh{}\PYGZsh{}\PYGZsh{}\PYGZsh{}\PYGZsh{}\PYGZsh{}\PYGZsh{}\PYGZsh{}\PYGZsh{}\PYGZsh{}\PYGZsh{}\PYGZsh{}\PYGZsh{}\PYGZsh{}\PYGZsh{}\PYGZsh{}\PYGZsh{}\PYGZsh{}\PYGZsh{}\PYGZsh{}\PYGZsh{}\PYGZsh{}\PYGZsh{}\PYGZsh{}\PYGZsh{}\PYGZsh{}\PYGZsh{}\PYGZsh{}\PYGZsh{}\PYGZsh{}\PYGZsh{}\PYGZsh{}
\PYGZsh{} Mol.  Name Chem.  Pot.  (K)
\PYGZsh{}\PYGZsh{}\PYGZsh{}\PYGZsh{}\PYGZsh{}\PYGZsh{}\PYGZsh{}\PYGZsh{}\PYGZsh{}\PYGZsh{}\PYGZsh{}\PYGZsh{}\PYGZsh{}\PYGZsh{}\PYGZsh{}\PYGZsh{}\PYGZsh{}\PYGZsh{}\PYGZsh{}\PYGZsh{}\PYGZsh{}\PYGZsh{}\PYGZsh{}\PYGZsh{}\PYGZsh{}\PYGZsh{}\PYGZsh{}\PYGZsh{}\PYGZsh{}\PYGZsh{}\PYGZsh{}\PYGZsh{}\PYGZsh{}
ChemPot AR \PYGZhy{}968
\end{sphinxVerbatim}

\item[{\sphinxcode{\sphinxupquote{Fugacity}}}] \leavevmode
For Grand Canonical (GC) ensemble runs only: Fugacity at which simulation is run.
\begin{itemize}
\item {} 
Value 1: String - The resname to apply this fugacity.

\item {} 
Value 2: Double - The fugacity value in bar.

\end{itemize}

\begin{sphinxadmonition}{note}{Note:}
For binary systems, include multiple copies of the tag (one per residue kind).
\end{sphinxadmonition}

\begin{sphinxadmonition}{note}{Note:}
If there is a molecule kind that cannot be transfer between boxes (in PDB file the beta value is set to 1.00 or 2.00) an arbitrary value e.g. 0.00 can be assigned to the resname.
\end{sphinxadmonition}

\fvset{hllines={, ,}}%
\begin{sphinxVerbatim}[commandchars=\\\{\}]
\PYGZsh{}\PYGZsh{}\PYGZsh{}\PYGZsh{}\PYGZsh{}\PYGZsh{}\PYGZsh{}\PYGZsh{}\PYGZsh{}\PYGZsh{}\PYGZsh{}\PYGZsh{}\PYGZsh{}\PYGZsh{}\PYGZsh{}\PYGZsh{}\PYGZsh{}\PYGZsh{}\PYGZsh{}\PYGZsh{}\PYGZsh{}\PYGZsh{}\PYGZsh{}\PYGZsh{}\PYGZsh{}\PYGZsh{}\PYGZsh{}\PYGZsh{}\PYGZsh{}\PYGZsh{}\PYGZsh{}\PYGZsh{}\PYGZsh{}
\PYGZsh{} Mol.  Name Fugacity (bar)
\PYGZsh{}\PYGZsh{}\PYGZsh{}\PYGZsh{}\PYGZsh{}\PYGZsh{}\PYGZsh{}\PYGZsh{}\PYGZsh{}\PYGZsh{}\PYGZsh{}\PYGZsh{}\PYGZsh{}\PYGZsh{}\PYGZsh{}\PYGZsh{}\PYGZsh{}\PYGZsh{}\PYGZsh{}\PYGZsh{}\PYGZsh{}\PYGZsh{}\PYGZsh{}\PYGZsh{}\PYGZsh{}\PYGZsh{}\PYGZsh{}\PYGZsh{}\PYGZsh{}\PYGZsh{}\PYGZsh{}\PYGZsh{}\PYGZsh{}
Fugacity AR 0.1
Fugacity Si 0.0
Fugacity O 0.0
\end{sphinxVerbatim}

\item[{\sphinxcode{\sphinxupquote{DisFreq}}}] \leavevmode
Fractional percentage at which displacement move will occur.
\begin{itemize}
\item {} 
Value 1: Double - \% Displacement

\end{itemize}

\item[{\sphinxcode{\sphinxupquote{RotFreq}}}] \leavevmode
Fractional percentage at which rigid rotation move will occur.
\begin{itemize}
\item {} 
Value 1: Double - \% Rotatation

\end{itemize}

\item[{\sphinxcode{\sphinxupquote{IntraSwapFreq}}}] \leavevmode
Fractional percentage at which molecule will be removed from a box and inserted into the same box using configurational bias algorithm.
\begin{itemize}
\item {} 
Value 1: Double - \% Intra molecule swap

\end{itemize}

\item[{\sphinxcode{\sphinxupquote{RegrowthFreq}}}] \leavevmode
Fractional percentage at which part of the molecule will be deleted and then regrown using configurational bias algorithm.
\begin{itemize}
\item {} 
Value 1: Double - \% Molecular growth

\end{itemize}

\item[{\sphinxcode{\sphinxupquote{VolFreq}}}] \leavevmode
For isobaric-isothermal ensemble and Gibbs ensemble runs only: Fractional percentage at which molecule will be removed from one box and inserted into the other box using configurational bias algorithm.
\begin{itemize}
\item {} 
Value 1: Double - \% Volume swaps

\end{itemize}

\item[{\sphinxcode{\sphinxupquote{SwapFreq}}}] \leavevmode
For Gibbs and Grand Canonical (GC) ensemble runs only: Fractional percentage at which molecule swap move will occur.
\begin{itemize}
\item {} 
Value 1: Double - \% Molecule swaps

\end{itemize}

\fvset{hllines={, ,}}%
\begin{sphinxVerbatim}[commandchars=\\\{\}]
\PYGZsh{}\PYGZsh{}\PYGZsh{}\PYGZsh{}\PYGZsh{}\PYGZsh{}\PYGZsh{}\PYGZsh{}\PYGZsh{}\PYGZsh{}\PYGZsh{}\PYGZsh{}\PYGZsh{}\PYGZsh{}\PYGZsh{}\PYGZsh{}\PYGZsh{}\PYGZsh{}\PYGZsh{}\PYGZsh{}\PYGZsh{}\PYGZsh{}\PYGZsh{}\PYGZsh{}\PYGZsh{}\PYGZsh{}\PYGZsh{}\PYGZsh{}\PYGZsh{}\PYGZsh{}\PYGZsh{}\PYGZsh{}\PYGZsh{}
\PYGZsh{} MOVE FREQEUNCY
\PYGZsh{}\PYGZsh{}\PYGZsh{}\PYGZsh{}\PYGZsh{}\PYGZsh{}\PYGZsh{}\PYGZsh{}\PYGZsh{}\PYGZsh{}\PYGZsh{}\PYGZsh{}\PYGZsh{}\PYGZsh{}\PYGZsh{}\PYGZsh{}\PYGZsh{}\PYGZsh{}\PYGZsh{}\PYGZsh{}\PYGZsh{}\PYGZsh{}\PYGZsh{}\PYGZsh{}\PYGZsh{}\PYGZsh{}\PYGZsh{}\PYGZsh{}\PYGZsh{}\PYGZsh{}\PYGZsh{}\PYGZsh{}\PYGZsh{}
DisFreq 0.49
RotFreq 0.10
VolFreq 0.01
SwapFreq 0.20
IntraSwapFreq 0.10
RegrowthFreq 0.10
\end{sphinxVerbatim}

\begin{sphinxadmonition}{note}{Note:}
All move percentages should add up to 1.0; otherwise, the program will terminate.
\end{sphinxadmonition}

\item[{\sphinxcode{\sphinxupquote{useConstantArea}}}] \leavevmode
For Isobaric-Isothermal ensemble and Gibbs ensemble runs only: Considers to change the volume of the simulation box by fixing the cross-sectional area (x-y plane).
\begin{itemize}
\item {} 
Value 1: Boolean - If true volume will change only in z axis, If false volume will change with constant axis ratio.

\end{itemize}

\begin{sphinxadmonition}{note}{Note:}
By default, useConstantArea will be set to false if no value was set. It means, the volume of the box will change in a way to maintain the constant axis ratio.
\end{sphinxadmonition}

\item[{\sphinxcode{\sphinxupquote{FixVolBox0}}}] \leavevmode
For adsorption simulation in NPT Gibbs ensemble runs only: Changing the volume of fluid phase (Box 1) to maintain the constant imposed pressure and temperature, while keeping the volume of adsorbed phase (Box 0) fix.
\begin{itemize}
\item {} 
Value 1: Boolean - If true volume of adsorbed phase will remain constant, If false volume of adsorbed phase will change.

\end{itemize}

\item[{\sphinxcode{\sphinxupquote{CellBasisVector}}}] \leavevmode
Defines the shape and size of the simulation periodic cell. \sphinxcode{\sphinxupquote{CellBasisVector1}}, \sphinxcode{\sphinxupquote{CellBasisVector2}}, \sphinxcode{\sphinxupquote{CellBasisVector3}} represent the cell basis vector \(a,b,c\), respectively. This tag may occur multiple times. It occurs once for NVT and NPT, but twice for Gibbs ensemble or GC ensemble.
\begin{itemize}
\item {} 
Value 1: Integer - Sets box number (first box is box ‘0’).

\item {} 
Value 2: Double - x value of cell basis vector in Angstroms.

\item {} 
Value 3: Double - y value of cell basis vector in Angstroms.

\item {} 
Value 4: Double - z value of cell basis vector in Angstroms.

\end{itemize}

\begin{sphinxadmonition}{note}{Note:}
If the number of defined boxes were not compatible to simulation type, the program will be terminated.
\end{sphinxadmonition}

Example for NVT and NPT ensemble. In this example, each vector is perpendicular to the other two (\(\alpha = 90, \beta = 90, \gamma = 90\)), as indicated by a single x, y, or z value being specified by each and making a rectangular 3-D box:

\fvset{hllines={, ,}}%
\begin{sphinxVerbatim}[commandchars=\\\{\}]
\PYGZsh{}\PYGZsh{}\PYGZsh{}\PYGZsh{}\PYGZsh{}\PYGZsh{}\PYGZsh{}\PYGZsh{}\PYGZsh{}\PYGZsh{}\PYGZsh{}\PYGZsh{}\PYGZsh{}\PYGZsh{}\PYGZsh{}\PYGZsh{}\PYGZsh{}\PYGZsh{}\PYGZsh{}\PYGZsh{}\PYGZsh{}\PYGZsh{}\PYGZsh{}\PYGZsh{}\PYGZsh{}\PYGZsh{}\PYGZsh{}\PYGZsh{}\PYGZsh{}\PYGZsh{}\PYGZsh{}\PYGZsh{}\PYGZsh{}
\PYGZsh{} BOX DIMENSION \PYGZsh{}, X, Y, Z
\PYGZsh{}\PYGZsh{}\PYGZsh{}\PYGZsh{}\PYGZsh{}\PYGZsh{}\PYGZsh{}\PYGZsh{}\PYGZsh{}\PYGZsh{}\PYGZsh{}\PYGZsh{}\PYGZsh{}\PYGZsh{}\PYGZsh{}\PYGZsh{}\PYGZsh{}\PYGZsh{}\PYGZsh{}\PYGZsh{}\PYGZsh{}\PYGZsh{}\PYGZsh{}\PYGZsh{}\PYGZsh{}\PYGZsh{}\PYGZsh{}\PYGZsh{}\PYGZsh{}\PYGZsh{}\PYGZsh{}\PYGZsh{}\PYGZsh{}
CellBasisVector1 0 40.00 00.00 00.00
CellBasisVector2 0 00.00 40.00 00.00
CellBasisVector3 0 00.00 00.00 80.00
\end{sphinxVerbatim}

Example for Gibbs ensemble and GC ensemble ensemble. In this example, In the first box, only vector a and c are perpendicular to each other (\(\alpha = 90, \beta = 90, \gamma = 120\)), and making a non-orthogonal simulation cell with the cell length \(a, b, c\) of 36.91, 39.91, and 76.98 Angstroms, respectively. In the second box, each vector is perpendicular to the other two (\(\alpha = 90, \beta = 90, \gamma = 90\)), as indicated by a single x, y, or z value being specified by each and making a cubic box:

\fvset{hllines={, ,}}%
\begin{sphinxVerbatim}[commandchars=\\\{\}]
\PYGZsh{}\PYGZsh{}\PYGZsh{}\PYGZsh{}\PYGZsh{}\PYGZsh{}\PYGZsh{}\PYGZsh{}\PYGZsh{}\PYGZsh{}\PYGZsh{}\PYGZsh{}\PYGZsh{}\PYGZsh{}\PYGZsh{}\PYGZsh{}\PYGZsh{}\PYGZsh{}\PYGZsh{}\PYGZsh{}\PYGZsh{}\PYGZsh{}\PYGZsh{}\PYGZsh{}\PYGZsh{}\PYGZsh{}\PYGZsh{}\PYGZsh{}\PYGZsh{}\PYGZsh{}\PYGZsh{}\PYGZsh{}\PYGZsh{}
\PYGZsh{} BOX DIMENSION \PYGZsh{}, X, Y, Z
\PYGZsh{}\PYGZsh{}\PYGZsh{}\PYGZsh{}\PYGZsh{}\PYGZsh{}\PYGZsh{}\PYGZsh{}\PYGZsh{}\PYGZsh{}\PYGZsh{}\PYGZsh{}\PYGZsh{}\PYGZsh{}\PYGZsh{}\PYGZsh{}\PYGZsh{}\PYGZsh{}\PYGZsh{}\PYGZsh{}\PYGZsh{}\PYGZsh{}\PYGZsh{}\PYGZsh{}\PYGZsh{}\PYGZsh{}\PYGZsh{}\PYGZsh{}\PYGZsh{}\PYGZsh{}\PYGZsh{}\PYGZsh{}\PYGZsh{}
CellBasisVector1 0 36.91 00.00 00.00
CellBasisVector2 0 \PYGZhy{}18.45 31.96 00.00
CellBasisVector3 0 00.00 00.00 76.98

CellBasisVector1 1 60.00 00.00 00.00
CellBasisVector2 1 00.00 60.00 00.00
CellBasisVector3 1 00.00 00.00 60.00
\end{sphinxVerbatim}

\begin{sphinxadmonition}{warning}{Warning:}
In case of \sphinxcode{\sphinxupquote{Restart true}}, box dimension does not need to be specified. If it is specified, program will read it but it will be ignored and replaced by the printed cell dimensions and angles in the restart PDB output file from GOMC (\sphinxcode{\sphinxupquote{OutputName\_BOX\_0\_restart.pdb}} and \sphinxcode{\sphinxupquote{Output\_Name\_BOX\_1\_restart.pdb}}).
\end{sphinxadmonition}

\item[{\sphinxcode{\sphinxupquote{CBMC\_First}}}] \leavevmode
Number of CBMC trials to choose the first atom position (Lennard-Jones trials for first seed growth).
\begin{itemize}
\item {} 
Value 1: Integer - Number of initial insertion sites to try.

\end{itemize}

\item[{\sphinxcode{\sphinxupquote{CBMC\_Nth}}}] \leavevmode
Number of CBMC trials to choose the later atom positions (Lennard-Jones trials for first seed growth).
\begin{itemize}
\item {} 
Value 1: Integer - Number of LJ trials for growing later atom positions.

\end{itemize}

\item[{\sphinxcode{\sphinxupquote{CBMC\_Ang}}}] \leavevmode
Number of CBMC bending angle trials to perform for geometry (per the coupled-decoupled CBMC scheme).
\begin{itemize}
\item {} 
Value 1: Integer - Number of trials per angle.

\end{itemize}

\item[{\sphinxcode{\sphinxupquote{CBMC\_Dih}}}] \leavevmode
Number of CBMC dihedral angle trials to perform for geometry (per the coupled-decoupled CBMC scheme).
\begin{itemize}
\item {} 
Value 1: Integer - Number of trials per dihedral.

\end{itemize}

\fvset{hllines={, ,}}%
\begin{sphinxVerbatim}[commandchars=\\\{\}]
\PYGZsh{}\PYGZsh{}\PYGZsh{}\PYGZsh{}\PYGZsh{}\PYGZsh{}\PYGZsh{}\PYGZsh{}\PYGZsh{}\PYGZsh{}\PYGZsh{}\PYGZsh{}\PYGZsh{}\PYGZsh{}\PYGZsh{}\PYGZsh{}\PYGZsh{}\PYGZsh{}\PYGZsh{}\PYGZsh{}\PYGZsh{}\PYGZsh{}\PYGZsh{}\PYGZsh{}\PYGZsh{}\PYGZsh{}\PYGZsh{}\PYGZsh{}\PYGZsh{}\PYGZsh{}\PYGZsh{}\PYGZsh{}\PYGZsh{}
\PYGZsh{} CBMC TRIALS
\PYGZsh{}\PYGZsh{}\PYGZsh{}\PYGZsh{}\PYGZsh{}\PYGZsh{}\PYGZsh{}\PYGZsh{}\PYGZsh{}\PYGZsh{}\PYGZsh{}\PYGZsh{}\PYGZsh{}\PYGZsh{}\PYGZsh{}\PYGZsh{}\PYGZsh{}\PYGZsh{}\PYGZsh{}\PYGZsh{}\PYGZsh{}\PYGZsh{}\PYGZsh{}\PYGZsh{}\PYGZsh{}\PYGZsh{}\PYGZsh{}\PYGZsh{}\PYGZsh{}\PYGZsh{}\PYGZsh{}\PYGZsh{}\PYGZsh{}
CBMC\PYGZus{}First 10
CBMC\PYGZus{}Nth 4
CBMC\PYGZus{}Ang 100
CBMC\PYGZus{}Dih 30
\end{sphinxVerbatim}

\end{description}


\subsection{Output Controls}
\label{\detokenize{input_file:output-controls}}
This section contains all the values that control output in the control file. For example, certain variables control the naming of files dumped of the block-averaged thermodynamic variables of interest, the PDB files, etc.
\begin{description}
\item[{\sphinxcode{\sphinxupquote{OutputName}}}] \leavevmode
Unique name for simulation used to name the block average, PDB, and PSF output files.
\begin{itemize}
\item {} 
Value 1: String - Unique phhrase to identify this system.

\end{itemize}

\fvset{hllines={, ,}}%
\begin{sphinxVerbatim}[commandchars=\\\{\}]
\PYGZsh{}\PYGZsh{}\PYGZsh{}\PYGZsh{}\PYGZsh{}\PYGZsh{}\PYGZsh{}\PYGZsh{}\PYGZsh{}\PYGZsh{}\PYGZsh{}\PYGZsh{}\PYGZsh{}\PYGZsh{}\PYGZsh{}\PYGZsh{}\PYGZsh{}\PYGZsh{}\PYGZsh{}\PYGZsh{}\PYGZsh{}\PYGZsh{}\PYGZsh{}\PYGZsh{}\PYGZsh{}\PYGZsh{}\PYGZsh{}\PYGZsh{}\PYGZsh{}\PYGZsh{}\PYGZsh{}\PYGZsh{}\PYGZsh{}
\PYGZsh{} OUTPUT FILE NAME
\PYGZsh{}\PYGZsh{}\PYGZsh{}\PYGZsh{}\PYGZsh{}\PYGZsh{}\PYGZsh{}\PYGZsh{}\PYGZsh{}\PYGZsh{}\PYGZsh{}\PYGZsh{}\PYGZsh{}\PYGZsh{}\PYGZsh{}\PYGZsh{}\PYGZsh{}\PYGZsh{}\PYGZsh{}\PYGZsh{}\PYGZsh{}\PYGZsh{}\PYGZsh{}\PYGZsh{}\PYGZsh{}\PYGZsh{}\PYGZsh{}\PYGZsh{}\PYGZsh{}\PYGZsh{}\PYGZsh{}\PYGZsh{}\PYGZsh{}
OutputName ISB T 270 K
\end{sphinxVerbatim}

\item[{\sphinxcode{\sphinxupquote{CoordinatesFreq}}}] \leavevmode
Controls output of PDB file (coordinates). If PDB dumping was enabled, one file for NVT or NPT and two files for Gibbs ensemble or GC ensemble will be dumped into \sphinxcode{\sphinxupquote{OutputName\_BOX\_n.pdb}}, where n defines the box number.
\begin{itemize}
\item {} 
Value 1: Boolean - “true” enables dumping these files; “false” disables dumping.

\item {} 
Value 2: Ulong - Steps per dump PDB frame. It should be less than or equal to RunSteps. If this keyword could not be found in configuration file, its value will be assigned a default value to dump 10 frames.

\end{itemize}

\begin{sphinxadmonition}{note}{Note:}
The PDB file contains an entry for every ATOM, in all boxes read. This allows VMD (which requires a constant number of atoms) to properly parse frames, with a bit of help. Atoms that are not currently in a specific box are given the coordinate (0.00, 0.00, 0.00). The occupancy value corresponds to the box a molecule is currently in (e.g. 0.00 for box 0; 1.00 for box 1).
\end{sphinxadmonition}

\begin{sphinxadmonition}{note}{Note:}
At the beginning of simulation, a merged PSF file will be dumped into OutputName merged.pdb, in which all boxes will be dumped. It also contains the topology for every molecule in both boxes, corresponding to the merged PDB format. Loading PDB files into merged PSF file in VMD allows the user to visualize and analyze the results. In addition, this file can be used to load into GOMC once restart simulation was active.
\end{sphinxadmonition}

\item[{\sphinxcode{\sphinxupquote{RestartFreq}}}] \leavevmode
Controls the output of the last state of simulation at a specified step in PDB files (coordinates) OutputName BOX n restart.pdb, where n defines the box number. Header part of this file contains important information and will be needed to restart the simulation:
\begin{itemize}
\item {} 
Simulation cell dimensions and angles.

\item {} 
Maximum amount of displacement (Å), rotation (\(\delta\)), and volume (\(Å^3\)) that used in Displacement, Rotation, and Volume move.

\end{itemize}

If PDB dumping was enabled, one file for NVT or NPT and two files for Gibbs ensemble or GC ensemble will be dumped.
\begin{itemize}
\item {} 
Value 1: Boolean - “true” enables dumping these files; “false” disables dumping.

\item {} 
Value 2: Ulong - Steps per dump last state of simulation to PDB files. It should be less than or equal to RunSteps. If this keyword could not be found in the configuration file, RestartFreq value will be assigned by default.

\end{itemize}

\begin{sphinxadmonition}{note}{Note:}
The restart PDB file contains only ATOM that exist in each boxes at specified steps. This allows the user to load this file into GOMC once restart simulation was active.
\end{sphinxadmonition}

\begin{sphinxadmonition}{note}{Note:}
CoordinatesFreq must be a common multiple of RestartFreq or vice versa.
\end{sphinxadmonition}

\item[{\sphinxcode{\sphinxupquote{ConsoleFreq}}}] \leavevmode
Controls the output to STDIO (“the console”) of messages such as acceptance statistics, and run timing info. In addition, instantaneously-selected thermodynamic properties will be output to this file.
\begin{itemize}
\item {} 
Value 1: Boolean - “true” enables message printing; “false” disables dumping.

\item {} 
Value 2: Ulong - Number of steps per print. If this keyword could not be found in the configuration file, the value will be assigned by default to dump 1000 output for RunSteps greater than 1000 steps and 100 output for RunSteps less than 1000 steps.

\end{itemize}

\item[{\sphinxcode{\sphinxupquote{BlockAverageFreq}}}] \leavevmode
Controls the block averages output of selected thermodynamic properties. Block averages are averages of thermodynamic values of interest for chunks of the simulation (for post-processing of averages or std. dev. in those values).
\begin{itemize}
\item {} 
Value 1: Boolean - “true” enables printing block average; “false” disables it.

\item {} 
Value 2: Ulong - Number of steps per block-average output file. If this keyword cannot be found in the configuration file, its value will be assigned a default to dump 100 output.

\end{itemize}

\item[{\sphinxcode{\sphinxupquote{HistogramFreq}}}] \leavevmode
Controls the histograms. Histograms are a binned listing of observation frequency for a specific thermodynamic variable. In this code, they also control the output of a file containing energy/molecule samples; it only will be used in GC ensemble simulations for histogram reweighting purposes.
\begin{itemize}
\item {} 
Value 1: Boolean - “true” enables printing histogram; “false” disables it.

\item {} 
Value 2: Ulong - Number of steps per histogram output file. If this keyword cannot be found in the configuration file, a value will be assigned by default to dump 1000 output for RunSteps greater than 1000 steps and 100 output for RunSteps less than 1000 steps.

\end{itemize}

\fvset{hllines={, ,}}%
\begin{sphinxVerbatim}[commandchars=\\\{\}]
\PYGZsh{}\PYGZsh{}\PYGZsh{}\PYGZsh{}\PYGZsh{}\PYGZsh{}\PYGZsh{}\PYGZsh{}\PYGZsh{}\PYGZsh{}\PYGZsh{}\PYGZsh{}\PYGZsh{}\PYGZsh{}\PYGZsh{}\PYGZsh{}\PYGZsh{}\PYGZsh{}\PYGZsh{}\PYGZsh{}\PYGZsh{}\PYGZsh{}\PYGZsh{}\PYGZsh{}\PYGZsh{}\PYGZsh{}\PYGZsh{}\PYGZsh{}\PYGZsh{}\PYGZsh{}\PYGZsh{}\PYGZsh{}\PYGZsh{}
\PYGZsh{} STATISTICS Enable, Freq.
\PYGZsh{}\PYGZsh{}\PYGZsh{}\PYGZsh{}\PYGZsh{}\PYGZsh{}\PYGZsh{}\PYGZsh{}\PYGZsh{}\PYGZsh{}\PYGZsh{}\PYGZsh{}\PYGZsh{}\PYGZsh{}\PYGZsh{}\PYGZsh{}\PYGZsh{}\PYGZsh{}\PYGZsh{}\PYGZsh{}\PYGZsh{}\PYGZsh{}\PYGZsh{}\PYGZsh{}\PYGZsh{}\PYGZsh{}\PYGZsh{}\PYGZsh{}\PYGZsh{}\PYGZsh{}\PYGZsh{}\PYGZsh{}\PYGZsh{}
CoordinatesFreq   true 10000000
RestartFreq       true 1000000
ConsoleFreq       true 100000
BlockAverageFreq  true 100000
HistogramFreq     true 10000
\end{sphinxVerbatim}

\end{description}

The next section controls the output of the energy/molecule sample file and the distribution file for molecule counts, commonly referred to as the “histogram” output. This section is only required if Grand Canonical ensemble simulation was used.
\begin{description}
\item[{\sphinxcode{\sphinxupquote{DistName}}}] \leavevmode
Sets short phrase to naming molecule distribution file.
\begin{itemize}
\item {} 
Value 1: String - Short phrase which will be combined with \sphinxstyleemphasis{RunNumber} and \sphinxstyleemphasis{RunLetter} to use in the name of the binned histogram for molecule distribution.

\end{itemize}

\item[{\sphinxcode{\sphinxupquote{HistName}}}] \leavevmode
Sets short phrase to naming energy sample file.
\begin{itemize}
\item {} 
Value 1: String - Short phrase, which will be combined with \sphinxstyleemphasis{RunNumber} and \sphinxstyleemphasis{RunLetter}, to use in the name of the energy/molecule count sample file.

\end{itemize}

\item[{\sphinxcode{\sphinxupquote{RunNumber}}}] \leavevmode
Sets a number, which is a part of \sphinxstyleemphasis{DistName} and \sphinxstyleemphasis{HistName} file name.
\begin{itemize}
\item {} 
Value 1: Uint \textendash{} Run number to be used in the above file names.

\end{itemize}

\item[{\sphinxcode{\sphinxupquote{RunLetter}}}] \leavevmode
Sets a letter, which is a part of \sphinxstyleemphasis{DistName} and \sphinxstyleemphasis{HistName} file name.
\begin{itemize}
\item {} 
Value 1: Character \textendash{} Run letter to be used in above file names.

\end{itemize}

\item[{\sphinxcode{\sphinxupquote{SampleFreq}}}] \leavevmode
Controls histogram sampling frequency.
\begin{itemize}
\item {} 
Value 1: Uint \textendash{} the number of steps per histogram sample.

\end{itemize}

\fvset{hllines={, ,}}%
\begin{sphinxVerbatim}[commandchars=\\\{\}]
\PYGZsh{}\PYGZsh{}\PYGZsh{}\PYGZsh{}\PYGZsh{}\PYGZsh{}\PYGZsh{}\PYGZsh{}\PYGZsh{}\PYGZsh{}\PYGZsh{}\PYGZsh{}\PYGZsh{}\PYGZsh{}\PYGZsh{}\PYGZsh{}\PYGZsh{}\PYGZsh{}\PYGZsh{}\PYGZsh{}\PYGZsh{}\PYGZsh{}\PYGZsh{}\PYGZsh{}\PYGZsh{}\PYGZsh{}\PYGZsh{}\PYGZsh{}\PYGZsh{}\PYGZsh{}\PYGZsh{}\PYGZsh{}\PYGZsh{}
\PYGZsh{} OutHistSettings
\PYGZsh{}\PYGZsh{}\PYGZsh{}\PYGZsh{}\PYGZsh{}\PYGZsh{}\PYGZsh{}\PYGZsh{}\PYGZsh{}\PYGZsh{}\PYGZsh{}\PYGZsh{}\PYGZsh{}\PYGZsh{}\PYGZsh{}\PYGZsh{}\PYGZsh{}\PYGZsh{}\PYGZsh{}\PYGZsh{}\PYGZsh{}\PYGZsh{}\PYGZsh{}\PYGZsh{}\PYGZsh{}\PYGZsh{}\PYGZsh{}\PYGZsh{}\PYGZsh{}\PYGZsh{}\PYGZsh{}\PYGZsh{}\PYGZsh{}
DistName   dis
HistName   his
RunNumber  5
RunLetter  a
SampleFreq 200
\end{sphinxVerbatim}

\item[{\sphinxcode{\sphinxupquote{OutEnergy, OutPressure, OutMolNumber, OutDensity, OutVolume, OutSurfaceTension}}}] \leavevmode
Enables/Disables for specific kinds of file output for tracked thermodynamic quantities
\begin{itemize}
\item {} 
Value 1: Boolean \textendash{} “true” enables message output of block averages via this tracked parameter (and in some cases such as entry, components); “false” disables it.

\item {} 
Value 2: Boolean \textendash{} “true” enables message output of a fluctuation into the console file via this tracked parameter (and in some cases, such as entry, components); “false” disables it.

\end{itemize}

The keywords are available for the following ensembles


\begin{savenotes}\sphinxattablestart
\centering
\begin{tabulary}{\linewidth}[t]{|T|T|T|T|}
\hline
\sphinxstyletheadfamily 
Keyword
&\sphinxstyletheadfamily 
NVT
&\sphinxstyletheadfamily 
NPT \& Gibbs
&\sphinxstyletheadfamily 
GC
\\
\hline
OutEnergy
&
\(\checkmark\)
&
\(\checkmark\)
&
\(\checkmark\)
\\
\hline
OutPressure
&
\(\checkmark\)
&
\(\checkmark\)
&
\(\checkmark\)
\\
\hline
OutMolNumber
&&
\(\checkmark\)
&
\(\checkmark\)
\\
\hline
OutDensity
&&
\(\checkmark\)
&
\(\checkmark\)
\\
\hline
OutVolume
&&
\(\checkmark\)
&
\(\checkmark\)
\\
\hline
OutSurfaceTension
&
\(\checkmark\)
&&\\
\hline
\end{tabulary}
\par
\sphinxattableend\end{savenotes}

Here is an example:

\fvset{hllines={, ,}}%
\begin{sphinxVerbatim}[commandchars=\\\{\}]
\PYGZsh{}\PYGZsh{}\PYGZsh{}\PYGZsh{}\PYGZsh{}\PYGZsh{}\PYGZsh{}\PYGZsh{}\PYGZsh{}\PYGZsh{}\PYGZsh{}\PYGZsh{}\PYGZsh{}\PYGZsh{}\PYGZsh{}\PYGZsh{}\PYGZsh{}\PYGZsh{}\PYGZsh{}\PYGZsh{}\PYGZsh{}\PYGZsh{}\PYGZsh{}\PYGZsh{}\PYGZsh{}\PYGZsh{}\PYGZsh{}\PYGZsh{}\PYGZsh{}\PYGZsh{}\PYGZsh{}\PYGZsh{}\PYGZsh{}
\PYGZsh{} ENABLE: BLK AVE., FLUC.
\PYGZsh{}\PYGZsh{}\PYGZsh{}\PYGZsh{}\PYGZsh{}\PYGZsh{}\PYGZsh{}\PYGZsh{}\PYGZsh{}\PYGZsh{}\PYGZsh{}\PYGZsh{}\PYGZsh{}\PYGZsh{}\PYGZsh{}\PYGZsh{}\PYGZsh{}\PYGZsh{}\PYGZsh{}\PYGZsh{}\PYGZsh{}\PYGZsh{}\PYGZsh{}\PYGZsh{}\PYGZsh{}\PYGZsh{}\PYGZsh{}\PYGZsh{}\PYGZsh{}\PYGZsh{}\PYGZsh{}\PYGZsh{}\PYGZsh{}
OutEnergy         true true
OutPressure       true true
OutMolNum         true true
OutDensity        true true
OutVolume         true true
OutSurfaceTention false false
\end{sphinxVerbatim}

\end{description}


\chapter{GOMC’s Output Files, Terminal Output}
\label{\detokenize{output_file:gomc-s-output-files-terminal-output}}\label{\detokenize{output_file::doc}}\begin{description}
\item[{GOMC currently supports several kinds of output:}] \leavevmode\begin{itemize}
\item {} 
STDIO (“console”) output

\item {} 
File output

\textendash{} PDB

\textendash{} PSF

\textendash{} Block Averages

\end{itemize}

\end{description}

GOMC output units:


\begin{savenotes}\sphinxattablestart
\centering
\begin{tabulary}{\linewidth}[t]{|T|T|}
\hline
\sphinxstyletheadfamily 
Properties
&\sphinxstyletheadfamily 
Units
\\
\hline
Energy
&
\(K\)
\\
\hline
Pressure, Pressure Tensor
&
bar
\\
\hline
Heat of vaporization
&
\(KJ/mol\)
\\
\hline
Volume
&
\(Å^3\)
\\
\hline
Density
&
\(kg/m^3\)
\\
\hline
Surface Tension
&
\(mN/m\)
\\
\hline
\end{tabulary}
\par
\sphinxattableend\end{savenotes}


\section{Console Output}
\label{\detokenize{output_file:console-output}}
A variety of useful information relating to instantaneous statistical and thermodynamic data (move trials, acceptance rates, file I/O messages warnings, and other kinds of information) is printed to the STDIO, which, in Linux, will typically be displayed in the terminal. This output can be redirected into a log file in Linux using the \sphinxcode{\sphinxupquote{\textgreater{}}} operator.

\fvset{hllines={, ,}}%
\begin{sphinxVerbatim}[commandchars=\\\{\}]
\PYGZdl{} GOMC CPU NVT in.conf \PYGZgt{} out isobutane.log \PYG{p}{\PYGZam{}}
\end{sphinxVerbatim}

Statistical and thermodynamic information is provided in console output.
\begin{itemize}
\item {} 
Energy

\textendash{} Intermolecular (LJ)

\textendash{} Intramolecular bonded

\textendash{} Intramolecular nonbonded

\textendash{} Tail corrections

\textendash{} Electrostatic real

\textendash{} Electrostatic Reciprocal

\textendash{} Electrostatic self

\textendash{} Electrostatic correction

\textendash{} Total electrostatic energy (sum of real, reciprocal, self, and correction) \textendash{} Total Energy (sum of the all energies)

\item {} 
Pressure, Pressure Tensor (\(P_{xx},P_{yy},P_{zz}\))

\item {} 
Volume

\item {} 
Total molecule number

\item {} 
Total Density

\item {} 
Surface Tension

\item {} 
Mole fraction of each species

\end{itemize}

Detailed move, energy, and statistical or thermodynamic information for each simulation box will be printed in three different sections. Each section’s title will start with \sphinxcode{\sphinxupquote{MTITLE}}, \sphinxcode{\sphinxupquote{ETITLE}}, and \sphinxcode{\sphinxupquote{STITLE}} for move, energy, and statistical information, respectively. The instantaneous values for each section will start with \sphinxcode{\sphinxupquote{MOVE\_\#}}, \sphinxcode{\sphinxupquote{ENER\_\#}}, and \sphinxcode{\sphinxupquote{STAT\_\#}} for move, energy, and statistical values, respectively. Where, \# is the simulation box number. In addition, if pressure calculation is activated and enabled to print, pressure tensor will be printed in the console output file. This section starts with \sphinxcode{\sphinxupquote{PRES\_\#}} and print the diagonal value of pressure tensor \(P_{xx}\), \(P_{yy}\),and \(P_{zz}\), respectively. The second element after the title of each section is the step number.

In order to extract the desired information from the console file, “grep” and “awk” commands can be used with a proper title section. For example, in order to extract total energy of the system, the following command needs to be executed in terminal:

\fvset{hllines={, ,}}%
\begin{sphinxVerbatim}[commandchars=\\\{\}]
\PYGZdl{} grep \PYG{l+s+s2}{\PYGZdq{}ENER\PYGZus{}0\PYGZdq{}} output console.log \PYG{p}{\textbar{}} awk \PYG{l+s+s1}{\PYGZsq{}\PYGZob{}print \PYGZdl{}3\PYGZcb{}\PYGZsq{}}
\end{sphinxVerbatim}

Here, “output\_console.log” is the console output file and “\$3” represents the second element of the “ENERGY\_BOX\_0” section.

\begin{sphinxadmonition}{note}{Note:}
\sphinxstyleemphasis{Surface Tension is calculated using Virial method according to following equation},
\begin{equation*}
\begin{split}\gamma = \frac{1}{2A_{xy}} \int_{0}^{L} \bigg(P_{zz} - \frac{P_{xx} + P_{yy}}{2} \bigg) dz\end{split}
\end{equation*}\end{sphinxadmonition}

The first section of this console output typically includes some information relating the system, CPU, GPU, and RAM. In continue, console output includes information regarding the input file (configuration file), force field reading, summary of the structure of the molecule, bonded and non-bonded parameters, and minimum and maximum coordinate of molecules. This output is important; it may contain text relating to issues encountered if there was an error in the current run (e.g. a bad parameter, unknown keyword, missing parameters in the configuration file, etc.)

\noindent\sphinxincludegraphics{{out1}.png}

\noindent\sphinxincludegraphics{{out2}.png}

\noindent\sphinxincludegraphics{{out3}.png}

Next, the energy and statistic title, initial energy and statistic of the system’s starting configuration will print:

\begin{sphinxadmonition}{note}{Note:}
If total energy of simulation is greater that \(1.0e^{12}\), System Total Energy Calculation will be performed at EqSteps to preserve energy value.
\end{sphinxadmonition}

\noindent\sphinxincludegraphics{{out4}.png}

After the simulation starts, move, energy, and statistical title, followed by their values for each simulation box, will print:

\noindent\sphinxincludegraphics{{out5}.png}

At the end of the run, timing information and other wrap up info will be printed.

\begin{sphinxadmonition}{note}{Note:}
Printed energy and statistical values are instantaneous values.
\end{sphinxadmonition}

\begin{sphinxadmonition}{note}{Note:}
In order to keep the format of console file consistent, if absolute value of calculated properties of simulation is greater that \(1.0e^{11}\), the value of 999999999 will be printed instead.
\end{sphinxadmonition}

\begin{sphinxadmonition}{note}{Note:}
Since mol fraction value is very small compare to other properties, 8 digit precision was used instead of 4 digit to print out the value.
\end{sphinxadmonition}

\begin{sphinxadmonition}{note}{Note:}
It’s important to watch the acceptance rates and adjust the move percentages and CBMC trial amounts to get the desired rate of move acceptance.
\end{sphinxadmonition}


\section{Block Output Files}
\label{\detokenize{output_file:block-output-files}}
GOMC tracks a number of thermodynamic variables of interest during the simulation and prints them all in one file for each box.
\begin{itemize}
\item {} 
Energy

\textendash{} Intermolecular (LJ)

\textendash{} Intramolecular bonded

\textendash{} Intramolecular nonbonded

\textendash{} Tail corrections

\textendash{} Electrostatic real

\textendash{} Electrostatic Reciprocal

\textendash{} Total Energy (sum of the all energies)

\item {} 
Virial

\item {} 
Pressure

\item {} 
Surface Tension (using virial method)

\item {} 
Volume

\item {} 
Total molecule number

\item {} 
Total Density

\item {} 
Mole fraction of each species

\item {} 
Heat of vaporization

\end{itemize}

At the beginning of each file, the title of each property followed by their average values is printed. Desired data can be extracted, as explained before, using the “awk” command. For example, in order to extract total density of the system, the following command need to be executed in terminal:

\fvset{hllines={, ,}}%
\begin{sphinxVerbatim}[commandchars=\\\{\}]
\PYGZdl{} cat Blk\PYGZus{}OutputName\PYGZus{}BOX\PYGZus{}0.dat \PYG{p}{\textbar{}} awk \PYG{l+s+s1}{\PYGZsq{}\PYGZob{}print \PYGZdl{}13\PYGZcb{}\PYGZsq{}}
\end{sphinxVerbatim}

Here, “Blk\_OutputName\_BOX\_0.dat” is the block-average file for simulation box 0 and “\$13” represents the 13th column of the block file.

\begin{sphinxadmonition}{note}{Note:}
In order to keep the format of console file consistent, if absolute value of calculated properties of simulation is greater that \(1.0e^{11}\), the value of 999999999 will be printed instead.
\end{sphinxadmonition}

\begin{sphinxadmonition}{note}{Note:}
Since mol fraction value is very small compare to other properties, 8 digit precision was used instead of 4 digit to print out the value.
\end{sphinxadmonition}


\section{Visualizing Simulation}
\label{\detokenize{output_file:visualizing-simulation}}
If \sphinxcode{\sphinxupquote{CoordinatesFreq}} is enabled in configuration file, GOMC will output the molecule coordinates every specified stpes. The PDB and PSF output (merging of atom entries) has already been mentioned/explained in previous sections. To recap: The PDB file’s ATOM entries’ occupancy is used to represent the box the molecule is in for the current frame. All molecules are listed in order in which they were read (i.e. if box 0 has \(1..N1\) molecules and box 1 has \(1..N2\) molecules, then all of the molecules in box 0 are listed first and all the molecules in box 1, i.e. \(1..N1\), \(N1 + 1..N1 + N2\)). PDB frames are written as standard PDBs to consecutive file frames.

To visualize, open the output PDB and PSF files by GOMC using VMD, type this command in the terminal:

For all simulation except Gibbs ensemble that has one simulation box:

\fvset{hllines={, ,}}%
\begin{sphinxVerbatim}[commandchars=\\\{\}]
\PYGZdl{} vmd ISB\PYGZus{}T\PYGZus{}270\PYGZus{}k\PYGZus{}merged.psf ISB\PYGZus{}T\PYGZus{}270\PYGZus{}k\PYGZus{}BOX\PYGZus{}0.pdb
\end{sphinxVerbatim}

For Gibbs ensemble, visualizing the first box:

\fvset{hllines={, ,}}%
\begin{sphinxVerbatim}[commandchars=\\\{\}]
\PYGZdl{} vmd ISB\PYGZus{}T\PYGZus{}270\PYGZus{}k\PYGZus{}merged.psf ISB\PYGZus{}T\PYGZus{}270\PYGZus{}k\PYGZus{}BOX\PYGZus{}0.pdb
\end{sphinxVerbatim}

For Gibbs ensemble, visualizing the second box:

\fvset{hllines={, ,}}%
\begin{sphinxVerbatim}[commandchars=\\\{\}]
\PYGZdl{} vmd ISB\PYGZus{}T\PYGZus{}270\PYGZus{}k\PYGZus{}merged.psf ISB\PYGZus{}T\PYGZus{}270\PYGZus{}k\PYGZus{}BOX\PYGZus{}1.pdb
\end{sphinxVerbatim}

\begin{sphinxadmonition}{note}{Note:}
Restart coordinate file (OutputName\_BOX\_0\_restart.pdb) cannot be visualize using merged psf file, because atom number does not match. However, you can still open it in vmd using following command and vmd will automatically find the bonds of the molecule based on the coordinates.
\end{sphinxadmonition}

\fvset{hllines={, ,}}%
\begin{sphinxVerbatim}[commandchars=\\\{\}]
\PYGZdl{} vmd ISB\PYGZus{}T\PYGZus{}270\PYGZus{}k\PYGZus{}BOX\PYGZus{}0\PYGZus{}restart.pdb
\end{sphinxVerbatim}


\chapter{Putting it all together: Running a GOMC Simulation}
\label{\detokenize{putting_all_together:putting-it-all-together-running-a-gomc-simulation}}\label{\detokenize{putting_all_together::doc}}
It is strongly recommended that you download the test system provided at \sphinxhref{http://gomc.eng.wayne.edu/downloads.html}{GOMC Website} or \sphinxhref{https://github.com/GOMC-WSU/GOMC\_Examples/tree/master}{Our Github Page}

Run different simulation types in order to become more familiar with different parameter and configuration files (*.conf).

To recap the previous examples, a simulation of isobutane will be completed for a single temperature point on the saturated vapor-liquid coexistence curve.

The general plan for running the simulation is:
\begin{enumerate}
\item {} 
Build GOMC (if not done already)

\item {} 
Copy GOMC executable to build directory

\item {} 
Create scripts, PDB, and topology file to build the system, plus in.dat file and parameter files to prepare for runtime

\item {} 
Build finished PDBs and PSFs using the simulation.

\item {} 
Run the simulation in the terminal.

\item {} 
Analyze the output.

\end{enumerate}

Please, complete steps 1 and 2; then, traverse to the directory, which should now contain a single file “GOMC\_CPU GEMC”. Next, six files need to be made:
\begin{itemize}
\item {} 
PDB file for isobutane

\item {} 
Topology file describing isobutane residue

\item {} 
Two \sphinxcode{\sphinxupquote{*.inp}} packmol scripts to pack two system boxes

\item {} 
Two \sphinxstyleemphasis{TCL} scripts to input into \sphinxcode{\sphinxupquote{PSFGen}} to generate the final configuration

\end{itemize}



\renewcommand{\indexname}{Index}
\printindex
\end{document}